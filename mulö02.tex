\documentclass[10pt,a3paper]{article}

\usepackage[margin=2cm]{geometry}

\usepackage{booktabs}
\usepackage{color}
\usepackage{soul}
\usepackage[linecolor=gray,backgroundcolor=yellow!50,textsize=tiny]{todonotes}
\usepackage[linguistics]{forest}
\usepackage{styles/my-gb4e-slides}

\usepackage{avm}

\usepackage[maxbibnames=99,
  maxcitenames=2,
  uniquelist=false,
  backend=biber,
  doi=false,
  url=false,
  isbn=false,
  bibstyle=biblatex-sp-unified,
  citestyle=sp-authoryear-comp]{biblatex}

\addbibresource{register.bib}

\newcommand{\eg}{e.\,g.}
\newcommand{\Eg}{E.\,g.}

\newcommand{\roland}[1]{\todo[backgroundcolor=green!30]{#1}}
\newcommand{\elizabeth}[1]{\todo[backgroundcolor=blue!30]{#1}}
\newcommand{\sarah}[1]{\todo[backgroundcolor=yellow!40]{#1}}
\newcommand{\felix}[1]{\todo[backgroundcolor=orange!30]{#1}}

\forestset{
  decide/.style={draw, chamfered rectangle, inner sep=2pt},
  finall/.style={rounded corners, fill=gray, text=white},
  intrme/.style={draw, rounded corners},
  yes/.style={edge label={node[near end, above, sloped, font=\scriptsize]{Ja}}},
  no/.style={edge label={node[near end, above, sloped, font=\scriptsize]{Nein}}}
}

\author{Roland Schäfer}
\title{HPSG: Musterlösung 2}
\date{\today}

\usepackage{fontspec}
\defaultfontfeatures{Ligatures=TeX,Numbers=OldStyle, Scale=MatchLowercase}
\setmainfont{Linux Libertine O}
\setsansfont{Linux Biolinum O}

\avmfont{\sc}
\avmsortfont{\it}
\avmvalfont{\it}

\pagestyle{empty}

\begin{document}

\maketitle

\thispagestyle{empty}

\section{Material}

\noindent Es geht hier ausschließlich um Satz (\ref{ex:satz}).

\begin{exe}
  \ex Ein Dozent leider unter der Hitze.\label{ex:satz}
\end{exe}

\section{Version ohne Semantik für den ersten Überblick}

\begin{avm}
  \[ 
    phon & \< \rm\it Ein Dozent leider unter der Hitze \> \\
    head & \@7 \\
    subcat & \<\> \\
    hd-dtr & \[
      phon & \< \rm\it leidet unter der Hitze \> \\
      head & \@7 \\
      subcat & \@5 \< \@6 \> \\
      hd-dtr & \[
        phon & \< \rm\it leidet \> \\
        head & \@7 \[ \asort{verb}
          vform & fin 
        \]\\
        subcat & \@5 $\oplus$ \< \@4 \> 
      \] \\
      nhd-dtr & \@4  \[
        phon & \< \rm\it unter der Hitze \> \\
        head & \@8 \\
        subcat & \<\> \\
        hd-dtr & \[
          phon & \< \rm\it unter \> \\
          head & \@8 \[ \asort{prep}
            pform & unter-dat 
          \]\\
          subcat & \< \@3 \> 
        \] \\
        nhd-dtr & \@3  \[
          phon & \< \rm\it der Hitze \> \\
          head & \@9 \\
          subcat & \<\> \\
          hd-dtr & \[
            phon & \< \rm\it Hitze\> \\
            head & \@9 \[ \asort{noun}
            case & \@{12} dat
            \] \\
            subcat & \< \@2 \>
          \]\\
          nhd-dtr & \@2  \[
            phon & \< \rm\it der \> \\
            head & \[ \asort{det}
              case & \@{12}
            \] \\
            subcat & \<\> 
          \]
        \] \\
      \]
      \] \\
    nhd-dtr & \@6  \[
      phon & \< \rm\it ein Dozent \> \\
      head & \@{10} \\
      subcat & \<\> \\
      hd-dtr & \[
        phon & \< \rm\it Dozent \> \\
        head & \@{10} \[ \asort{noun}
          case & \@{11} nom \\
        \] \\
        subcat & \< \@1 \>
      \] \\
      nhd-dtr & \@1  \[
        phon & \< \rm\it ein \> \\
        head &\[ \asort{det}
          case & \@{11}
        \] \\
        subcat & \<\>
      \]
    \]
  \]
\end{avm}\\


\noindent Hinweis: Alle Strukturen mit Töchtern seien hier vom Typ \textit{hd-arg-phrase}.


\newpage

\section{Version mit Semantik}

\scalebox{0.95}{%
\begin{avm}
  \[ 
    phon & \< \rm\it Ein Dozent leider unter der Hitze \> \\
    head & \@7 \\
    subcat & \<\> \\
    cont & \@{17} \\
    hd-dtr & \[
      phon & \< \rm\it leidet unter der Hitze \> \\
      head & \@7 \\
      subcat & \< \@6 \> \\
      cont & \@{17} \\
      hd-dtr & \[
        phon & \< \rm\it leidet \> \\
        head & \@7 \[ \asort{verb}
          vform & fin 
        \]\\
        subcat & \<
          \@6 \[
            cont & \[
              index & \@{19} \\
              restr & \@{16}
          \] \],
          \@4 \[
          cont & \[
            index & \@{18} \\
            restr & \@{15}
          \] \]
          \> \\
        cont & \@{17} \[ 
          restr & \< \[ \asort{leiden-unter-rel}
            exp & \@{19} \\
            source & \@{18}
          \] \> $\oplus$ \@{15} $\oplus$ \@{16}
        \]
      \] \\
      nhd-dtr & \@4  \[
        phon & \< \rm\it unter der Hitze \> \\
        head & \@8 \\
        subcat & \<\> \\
        cont & \@{21} \\
        hd-dtr & \[
          phon & \< \rm\it unter \> \\
          head & \@8 \[ \asort{prep}
            pform & unter-dat 
          \]\\
          subcat & \< \@3 \[cont & \@{21} \> 
          \]\\
          cont & \@{21} 
        \] \\
        nhd-dtr & \@3  \[
          phon & \< \rm\it der Hitze \> \\
          head & \@9 \\
          subcat & \<\> \\
          cont & \@{21} \\ 
          hd-dtr & \[
            phon & \< \rm\it Hitze\> \\
            head & \@9 \[ \asort{noun}
            case & \@{12} dat
            \] \\
            subcat & \< \@2 \> \\
            cont & \@{21} \[
              index & \@{18} \[ \asort{index}
                per & 3 \\
                num & sg \\
                gen & f
              \] \\
              restr & \@{15} \< \[ \asort{hitze-rel}
                instance & \@{18}
              \] \>
            \] 
          \]\\
          nhd-dtr & \@2  \[
            phon & \< \rm\it der \> \\
            head & \[ \asort{det}
              case & \@{12}
            \] \\
            subcat & \<\> 
          \]
        \] \\
      \]
      \] \\
    nhd-dtr & \@6  \[
      phon & \< \rm\it ein Dozent \> \\
      head & \@{10} \\
      subcat & \<\> \\
      cont & \@{20} \\
      hd-dtr & \[
        phon & \< \rm\it Dozent \> \\
        head & \@{10} \[ \asort{noun}
          case & \@{11} nom \\
        \] \\
        subcat & \< \@1 \> \\
        cont & \@{20} \[ 
          index & \@{19} \[ \asort{index}
                per & 3 \\
                num & sg \\
                gen & m
              \]\\
          restr & \@{16} \< \[ \asort{dozent-rel}
            instance & \@{19}
          \] \>
        \]
      \] \\
      nhd-dtr & \@1  \[
        phon & \< \rm\it ein \> \\
        head &\[ \asort{det}
          case & \@{11}
        \] \\
        subcat & \<\>
      \]
    \]
  \]
\end{avm}%
}

\newpage


\subsection{Semantischer Beitrag des gesamten Satzes}

\noindent Wenn wir die Strukturteilungen auflösen, ergibt sich die folgende Semantik für den Satz.
Hier werden alle Strukturteilungsnummern aus der Analyse übernommen, um einen Abgleich zu ermöglichen.\\

\noindent\begin{avm}
  \[
    phon & \< \rm\it Ein Dozent leidet unter der Hitze \> \\
    cont & \@{17} \[
      index & \<\> \\
      restr & \< \[ \asort{leiden-unter-rel}
        exp & \@{19} \\
        source & \@{18}
      \] \> $\oplus$ 
      \@{15} \< \[ \asort{hitze-rel}
                instance & \@{18}
              \] \>
      $\oplus$
      \@{16} \< \[ \asort{dozent-rel}
            instance & \@{19}
          \] \>
    \]
  \]
\end{avm}\\

\vspace{\baselineskip}

\noindent Das ist äquivalent zu folgender Struktur (mit vereinfachter Darstellung der Liste und ohne unnötige Nummern):\\

\noindent\begin{avm}
  \[
    phon & \< \rm\it Ein Dozent leidet unter der Hitze \> \\
    cont & \[
      index & \<\> \\
      restr & \< \[ \asort{leiden-unter-rel}
        exp & \@{19} \\
        source & \@{18}
      \],
      \[ \asort{hitze-rel}
                instance & \@{18}
              \],
      \[ \asort{dozent-rel}
            instance & \@{19}
          \] \>
    \]
  \]
\end{avm}\\

\vspace{\baselineskip}

\noindent Die Indices sind verfügbar, weil sie über die Strukturteilung unter \textsc{restr} mitgenommen wurden. Sie müssen nicht nach oben zur Satzebene (in Form einer \textsc{indices}-Liste oder sowas) mitgeschleppt werden:\\

\noindent\begin{avm}
  \[
    phon & \< \rm\it Ein Dozent leidet unter der Hitze \> \\
    cont & \[
      index & \<\> \\
      restr & \< \[ \asort{leiden-unter-rel}
        exp & \@{19} \\
        source & \@{18}
      \],
      \[ \asort{hitze-rel}
              instance & \@{18} \[ \asort{index}
                per & 3 \\
                num & sg \\
                gen & f
              \]
              \],
      \[ \asort{dozent-rel}
            instance & \@{19} \[ \asort{index}
                per & 3 \\
                num & sg \\
                gen & m
              \]
          \] \>
    \]
  \]
\end{avm}\\


\noindent Man kann das Lesen als \textit{Es gibt ein Individuum (19), das leidet und ein Dozent ist, und es gibt ein weiteres Individuum (18), das dieses Leiden auslöst und (eine) Hitze ist.}
Im Wesentlichen ist das auch tatsächlich die Bedeutung des Satzes.\\

\noindent Achtung! Diese Semantik funktioniert nicht wirklich, wenn man den Beitrag von Quantoren (\textit{drei}, \textit{alle}, \textit{die meisten} usw.) modellieren will.
Beachten Sie in diesem Zusammenhang, dass die Determinierer hier keinerlei semantischen Beitrag leisten, was eigentlich unerwünscht ist.
Darüber sprechen wir dann gegen Ende des Semesters nochmal.


\subsection{Lexikoneintrag für das Verb}

\noindent Das ist nichts Anderes als oben in der großen Struktur, nur zuzüglich der Informationen über Kategorie (\textit{noun}, \textit{prep}), phrasalen Status (\textsc{subcat} $<>$) und Kasus bzw.\ \textsc{pform}.
Die Indices werden über die \textsc{subcat} mit denen der \textit{leiden-unter-rel} identifiziert.
Die Semantik holt sich zusätzlich (anders als im Buch) die \textsc{restr}-Listen der Argumente und gibt sie dann über das HFP weiter.\\

\begin{avm}
  \[
        phon & \< \rm\it leidet \> \\
        head & \[ \asort{verb}
          vform & fin 
        \]\\
        subcat & \<
          \[
            head & \[ \asort{noun}
              case & nom
            \] \\
            subcat & \<\> \\
            cont & \[
              index & \@{19} \\
              restr & \@{16}
          \] \],
        \[
          head & \[ \asort{prep}
            pform & unter-dat 
          \] \\
          subcat & \<\> \\
          cont & \[
            ind & \@{18} \\
            restr & \@{15}
          \] \]
          \> \\
        cont & \[
          restr & \< \[ \asort{leiden-unter-rel}
            exp & \@{19} \\
            source & \@{18}
          \] \> $\oplus$ \@{15} $\oplus$ \@{16}
        \]
   \]
\end{avm}


\subsection{Lexikoneintrag der Präposition}

\noindent Argument-markierende Präpositionen wie \textit{unter} in diesem Fall haben keine eigene Semantik und kopieren die Semantik des Arguments.
Diese wird dann über das HFP weitergegeben.
Solche Präpositionen heißen im Englischen auch \textit{case-marking prepositions}, weil sie -- wie gesagt -- keine eigene Präpositionalsemantik mitbringen, sondern wie ein verbregierter Kasus fungieren.
Das geht auch etwas über die Beispiele im Buch hinaus.\\

\begin{avm}
  \[
    phon & \< \rm\it unter \> \\
    head & \[ \asort{prep}
      pform & unter-dat 
    \]\\
    subcat & \< \[
      head & \[ \asort{noun}
        case & dat
      \]\\
      subcat & \<\> \\
      cont & \@{21}
    \] \> \\
    cont & \@{21} 
  \]
\end{avm}

\newpage

\subsection{Beispieleintrag für ein Substantiv}

\noindent Hier gibt es eigentlich nichts weiter zu beachten.\\

\begin{avm}
  \[
    phon & \< \rm\it Dozent \> \\
    head & \[ \asort{noun}
      case & \@{11} nom \\
    \] \\
    subcat & \< \[ head & \[
      \asort{det} 
      case & \@{11} 
    \] \] \> \\
    cont & \[
      index & \@{19} \\
      restr & \< \[ \asort{dozent-rel}
        instance & \@{19}
      \] \>
    \]
  \]
\end{avm}

\subsection{Schemata}

\textit{hd-arg-phrase} $\Rightarrow$
\begin{avm}
  \[
    subcat & \@1 \\
    hd-dtr|subcat & \@1 $\oplus$ \< \@2 \> \\
    nhd-dtr & \@2
  \]
\end{avm}


\subsection{Prinzipien}

\paragraph{Head Feature Principle (HFP)} regelt die Weitergabe der Kopfinformationen (Wortart, Kasus usw.) vom Kopf an die Phrase. 
\textit{hd-arg-phrase} sei ein Untertyp von \textit{hd-arg-phrase}.\\

\noindent \textit{headed-phrase} $\Rightarrow$
\begin{avm}
  \[
    head & \@1 \\
    hd-dtr|head & \@1
  \]
\end{avm}

\paragraph{Semantics Principle (SP)} regelt die Weitergabe der Semantik.
Hier geht es erstmal nur um Kopf-Komplement-Strukturen, in denen der Kopf die Semantik beisteuert.
Für Adjunktionsstrukturen bekommen wir dann einen zweiten Teil dieses Prinzips.\\

\noindent \textit{headed-phrase} $\Rightarrow$
\begin{avm}
  \[
    cont & \@1 \\
    hd-dtr|cont & \@1
  \]
\end{avm}

\section{Aufgaben}

\noindent Finden Sie in der Analyse jeweils die folgenden Numerierungen für Strukturteilung und erklären Sie, warum sie an mehreren Stellen stehen.
\textit{Erklären} heißt hier nur, das Schema, das Prinzip und\slash oder den Lexikoneintrag zu benennen, die es erlauben, die jeweilige Strukturteilung anzunehmen.
Geben Sie zusätzlich an, in welchen Lexikoneinträgen die konkreten Merkmale hinter den Numerierungen explizit genannt werden.
(Das können jeweils mehrere sein.)\\

\noindent Es kommen wirklich nur die hier genannten Arten von Einträgen, Schemata und Prinzipien zum Einsatz.
Diese erzeugen bereits die beträchtliche Komplexität der Analyse.\\

\noindent Die zu erklärenden Strukturteilungen werden hier schon thematisch gruppiert.

\begin{enumerate}
  \item 7, 8, 9, 10
  \item 1, 2, 3, 4, 6
  \item 18, 19
  \item 15, 16
  \item 11, 12
  \item 17, 20, 21
\end{enumerate}

\end{document}


