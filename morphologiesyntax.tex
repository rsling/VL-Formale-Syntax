\documentclass[10pt,a4paper]{article}

\usepackage[margin=2cm]{geometry}

\usepackage{booktabs}
\usepackage{color}
\usepackage{soul}
\usepackage[linecolor=gray,backgroundcolor=yellow!50,textsize=tiny]{todonotes}
\usepackage[linguistics]{forest}
\usepackage{styles/my-gb4e-slides}

\usepackage{avm}

\usepackage[maxbibnames=99,
  maxcitenames=2,
  uniquelist=false,
  backend=biber,
  doi=false,
  url=false,
  isbn=false,
  bibstyle=biblatex-sp-unified,
  citestyle=sp-authoryear-comp]{biblatex}

\addbibresource{register.bib}

\newcommand{\eg}{e.\,g.}
\newcommand{\Eg}{E.\,g.}

\definecolor{rot}{rgb}{0.7,0.2,0.0}
\newcommand{\rot}[1]{\textcolor{rot}{#1}}
\definecolor{blau}{rgb}{0.1,0.2,0.7}
\newcommand{\blau}[1]{\textcolor{blau}{#1}}
\definecolor{gruen}{rgb}{0.0,0.7,0.2}
\newcommand{\gruen}[1]{\textcolor{gruen}{#1}}
\definecolor{grau}{rgb}{0.6,0.6,0.6}
\newcommand{\grau}[1]{\textcolor{grau}{#1}}
\definecolor{orongsch}{RGB}{255,165,0}
\newcommand{\orongsch}[1]{\textcolor{orongsch}{#1}}
\definecolor{tuerkis}{RGB}{63,136,143}
\definecolor{braun}{RGB}{108,71,65}
\newcommand{\tuerkis}[1]{\textcolor{tuerkis}{#1}}
\newcommand{\braun}[1]{\textcolor{braun}{#1}}

\newcommand*{\mybox}[1]{\framebox{#1}}

\forestset{
  decide/.style={draw, chamfered rectangle, inner sep=2pt},
  finall/.style={rounded corners, fill=gray, text=white},
  intrme/.style={draw, rounded corners},
  yes/.style={edge label={node[near end, above, sloped, font=\scriptsize]{Ja}}},
  no/.style={edge label={node[near end, above, sloped, font=\scriptsize]{Nein}}}
}

\author{Roland Schäfer}
\title{HPSG: Morphologie vs.\ Syntax und Nullartikel}
\date{\today}

\usepackage{fontspec}
\defaultfontfeatures{Ligatures=TeX,Numbers=OldStyle, Scale=MatchLowercase}
\setmainfont{Linux Libertine O}
\setsansfont{Linux Biolinum O}

\avmfont{\sc}
\avmsortfont{\it}
\avmvalfont{\it}

\pagestyle{empty}

\begin{document}

\maketitle

\thispagestyle{empty}

\section{Problemstellung}

In HPSG verwenden Phonologie, Morphologie, Syntax, Semantik, Pragmatik usw.\ denselben Formalismus.
Insbesondere die Trennung zwischen Morphologie und Syntax ist daher eine relativ willkürliche in HPSG.
Hier wird gezeigt, wie man in HPSG mit der Tatsache umgehen kann, dass bestimmte Nomina (hier: Stoffnomina wie \textit{Zement}) auch im Singular ohne Determinierer auftreten können, in bestimmten Fällen aber doch einen Determinierer verlangen.
Wir beginnen mit der Annahme, dass diese Nomina im Lexikon eine leere \textsc{subcat} haben (Abschnitt~\ref{sec:lex}).
Es wird zunächst eine morphologische Lösung mit einer DLR vorgeschlagen (Abschnitt~\ref{sec:dlr}), dann eine äquivalente syntaktische mit einer unären Phrase (Abschnitt~\ref{sec:unaer}).
Zum Schluss wird eine syntaktische Lösung mit einem leeren Element (Nullartikel) gezeigt (Abschnitt~\ref{sec:null}), die einen modifizierten lexikalischen Eintrag des Nomens erfordert.
Die DLR-Lösung und die unäre Projektion sind auch komputational äquivalent, die Lösung mit einem leeren Element ist komputational erheblich schlechter.\\

\noindent Die Lösungen hier verstehen sich als Vereinfachungen.
Echte Modellierungen sind meistens erheblich komplexer.

\section{Lexikoneintrag}\label{sec:lex}

Der Typ \textit{mass-noun} sei ein Unter- oder Nebentyp von \textit{noun} mit leerer \textsc{subcat} und Stoffsemantik.
Den Typ \textit{mass-noun} brauchen wir vor allem, um die Anwendbarkeit der Regeln auf Stoffnomina zu beschränken.\\

\begin{avm}
  \[ \asort{word}
    phon & \<\rm\it Zement\> \\
    cat & \[
      head & \[ \asort{mass-noun} \] \\
      subcat & \<\> \\
    \] \\
    cont & \[
      index & \@1 \\
      restr & \<\[ \asort{cement-substance-rel}
        instance & \@1 \\
      \]\> \\
    \]
  \]
\end{avm}
 
\newpage

\section{Lösungsstrategien}

\subsection{Morphologie mit DLR}\label{sec:dlr}

Diese Regel macht aus einem Stoffnomen ein sortales Nomen (\textit{Wir liefern drei verschiedene Zemente.}).
Eine ähnliche Regel mit anderer Semantik könnte man für Portionsnomina erstellen (\textit{Wir hätten gerne drei Wasser.}).
Dass wir ein \textsc{psoa-arg} nehmen, liegt daran, dass hier die Extension des Nomens jenseits einer einfachen Teilmengenbildung verändert wird.
Wenn Ihnen das nicht klar ist, machen Sie sich keine weiteren Gedanken.
Dazu braucht man eigentlich verschärfte Semantikkenntnisse.\\

\begin{avm}
  \[ \asort{mass-sortal-noun-lex-rule}
    cat & \[
      head & \[ \asort{noun} \] \\
      subcat & \<\[ cat|head & \[ \asort{det} \] \]\> \\
    \]  \\
    cont & \[
      restr & \< \[ \asort{sort-of-mass-rel}
        psoa-arg & \@3 \\
      \]\> \\
    \] \\
    lex-dtr & \[
      cat & \[
        head & \[ \asort{mass-noun} \] \\
      \] \\
      cont &  \@3 \\
    \]\\
  \]
\end{avm}\\

\noindent Hinweis: In lexikalischen Regeln wird per Definition all das implizit aus der \textsc{lex-dtr} übernommen, über das nichts explizit ausgesagt wird.
Daher müssen wir hier nichts über \textsc{phon} usw. aussagen.

\subsection{Syntax mit unärer Projektion}\label{sec:unaer}

Für eine syntaktische \textit{unary-phrase} nehmen wir an, dass im Obertyp \textit{unary-phrase} bereits festgelegt wurde, dass \textsc{phon} und \textsc{cat|head} der Phrase gleich dem der \textsc{single-dtr} sind.\\

\noindent\textit{unary-phrase}~$\Rightarrow$\\
\begin{avm}
  \[ 
    phon & \@1 \\
    cat|head & \@2 \\
    single-dtr & \[
      phon & \@1 \\
      cat|head & \@2 \\
    \]\\
  \]
\end{avm}\\

\noindent Der Index muss explizit gesetzt werden, weil es keine Default-Übernahme-Regel für alle nicht erwähnten Merkmale gibt wie bei morphologischen Regeln.
Im Prinzip ist das aber genau dasselbe.\\

\noindent\textit{mass-sortal-noun-unary-phrase}~$\Rightarrow$\\
\begin{avm}
  \[ 
    cat & \[
      subcat & \<\[ cat|head & \[ \asort{det} \] \]\> \\
    \]  \\
    cont & \[
      index & \@1 \\
      restr & \< \[ \asort{sort-of-mass-rel}
        psoa-arg & \@3 \\
      \]\> \\
    \] \\
    single-dtr & \[
      cat & \[
        head & \[ \asort{mass-noun} \] \\
      \] \\
      cont &  \@3 \[
        index & \@1 \\
      \]\\
    \]\\
  \]
\end{avm}\\

\noindent Hinweis: Diese Regel ist technisch gesehen eine Implikation, die für einen bestimmten Typ unärer Phrasen (\textit{mass-sortal-noun-unary-phrase}) gilt.
Die DLR (\textit{mass-sortal-noun-lex-rule}) ist selber ein Zeichen, keine Implikation.

\newpage

\subsection{Syntax mit Nullartikel}\label{sec:null}

Der Nullartikel könnte wiefolgt aussehen.
Über \textsc{spec} sorgen wir dafür, dass er sich nur mit Stoffnomina kombinieren lässt.
Der Determinierer und das Nomen selegieren einander sozusagen (mutuelle Selektion).
Ansonsten macht der Nullartikel nichts.
Das Nomen selber hätte hier dann anders als in Abschnitt~\ref{sec:lex} eine normale \textsc{subcat}.\\

\begin{avm}
  \[
    phon & \<\> \\
    cat|head & \[ \asort{det}
      spec|cat|head & \[ \asort{mass-noun} \] \\
    \] \\
  \]
\end{avm}\\

\noindent Das ergibt dann NPs wie die folgende:\\

\begin{avm}
  \[ 
    phon & \<\rm\it Zement\> \\
    cat & \[
      head & \@1 \\
      subcat & \<\> \\
    \] \\
    hd-dtr & \[
      phon & \<\rm\it Zement\> \\
      cat & \[
        head & \@1  \\
        subcat & \< \@2 \[ cat|head & \[ \asort{det} \] \] \> \\
      \] \\
    \] \\
    non-hd-dtr & \@2 \[
      phon & \<\> \\
      cat & \[
        head & \[ \asort{det}
          spec|cat|head & \@1 \[ \asort{mass-noun} \] \\
        \]
      \] \\
    \]
  \]
\end{avm}\\

\noindent Diese Lösung ist zusätzlich zu ihrer komputationalen Inferiorität etwas ungünstig, weil wir noch weitere Regeln brauchen, um die Semantik von Stoffnomina mit Determinierer zu modellieren (sortale Lesart und Portionslesart).
Ansonsten würden sich normale Artikel mit den Stoffnomina verbinden, und wir könnten die besondere Semantik dabei nicht erzwingen.
Es wäre eine alternative (aber unintuitive) Möglichkeit, Stoffnomina lexikalisch als sortal zu definieren und den Nullartikel zu verwenden, um dann echte Stoffnomina aus ihnen zu machen.

\end{document}

