\documentclass[10pt,a3paper]{article}

\usepackage[margin=2cm]{geometry}

\usepackage{booktabs}
\usepackage{color}
\usepackage{soul}
\usepackage[linecolor=gray,backgroundcolor=yellow!50,textsize=tiny]{todonotes}
\usepackage[linguistics]{forest}
\usepackage{styles/my-gb4e-slides}

\usepackage{avm}

\usepackage[maxbibnames=99,
  maxcitenames=2,
  uniquelist=false,
  backend=biber,
  doi=false,
  url=false,
  isbn=false,
  bibstyle=biblatex-sp-unified,
  citestyle=sp-authoryear-comp]{biblatex}

\addbibresource{register.bib}

\newcommand{\eg}{e.\,g.}
\newcommand{\Eg}{E.\,g.}

\definecolor{rot}{rgb}{0,0,0}
\definecolor{blau}{rgb}{0,0,0}
\definecolor{gruen}{rgb}{0,0,0}
\definecolor{grau}{rgb}{0,0,0}
\definecolor{orongsch}{RGB}{0,0,0}
\definecolor{tuerkis}{RGB}{0,0,0}
\definecolor{braun}{RGB}{0,0,0}
\definecolor{pinku}{RGB}{0,0,0}
\definecolor{violett}{RGB}{0,0,0}

\newcommand{\rot}[1]{\textcolor{rot}{#1}}
\newcommand{\blau}[1]{\textcolor{blau}{#1}}
\newcommand{\gruen}[1]{\textcolor{gruen}{#1}}
\newcommand{\grau}[1]{\textcolor{grau}{#1}}
\newcommand{\orongsch}[1]{\textcolor{orongsch}{#1}}
\newcommand{\tuerkis}[1]{\textcolor{tuerkis}{#1}}
\newcommand{\braun}[1]{\textcolor{braun}{#1}}
\newcommand{\pinku}[1]{\textcolor{pinku}{#1}}
\newcommand{\violett}[1]{\textcolor{violett}{#1}}
\newcommand*{\mybox}[1]{\framebox{#1}}

\forestset{
  decide/.style={draw, chamfered rectangle, inner sep=2pt},
  finall/.style={rounded corners, fill=gray, text=white},
  intrme/.style={draw, rounded corners},
  yes/.style={edge label={node[near end, above, sloped, font=\scriptsize]{Ja}}},
  no/.style={edge label={node[near end, above, sloped, font=\scriptsize]{Nein}}}
}

\newcommand{\Sub}[1]{\ensuremath{_{\text{#1}}}}
\newcommand{\Up}[1]{\ensuremath{^{\text{#1}}}}
\newcommand{\UpSub}[2]{\ensuremath{^{\text{#1}}_{\text{#2}}}}
\newcommand{\Zeile}{\vspace{\baselineskip}}
\newcommand{\Halbzeile}{\vspace{0.5\baselineskip}}
\newcommand{\Viertelzeile}{\vspace{0.25\baselineskip}}


\author{Roland Schäfer}
\title{HPSG | Mündliche Prüfung | Bereich 5 \textit{Fernabhängigkeiten\slash V2-Sätze}}
\date{\today}

\usepackage{fontspec}
\defaultfontfeatures{Ligatures=TeX,Numbers=OldStyle, Scale=MatchLowercase}
\setmainfont{Linux Libertine O}
\setsansfont{Linux Biolinum O}

\avmfont{\sc}
\avmsortfont{\it}
\avmvalfont{\it}

\pagestyle{empty}

\setlength\parindent{0pt}

\begin{document}

\maketitle

\thispagestyle{empty}

\section{Material}


\begin{exe}
  \ex Kiki liest Die Strudlhofstiege.\label{ex:satz}
\end{exe}


\section{Lexikoneinträge}\label{sec:lex}

\begin{avm}
  \[ \asort{word}
    phon & \<\rm\it Kiki $\vee$ Die Strudlhofstiege\> \\
    loc|cat & \[
      head & \[ \asort{noun}
        cas & nom $\vee$ acc\\
        num & sg \\
      \]\\
      subcat & \<\> \\
    \] \\
  \]
\end{avm}

\begin{avm}
  \[ \asort{word}
    phon & \<\rm\it liest\> \\
    loc|cat & \[
      head & \[ \asort{verb}
        vform & fin \\
        initial & $-$ \\
      \]\\
      subcat & \<\[
        loc|cat & \[
          head & \[ \asort{noun}
            cas & nom \\
            num & sg \\
          \]\\
          subcat & \<\> \\
        \] \\
      \],
      \[
        loc|cat & \[
          head & \[ \asort{noun}
            cas & acc \\
          \]\\
          subcat & \<\> \\
        \] \\
      \]
      \> \\
    \] \\
  \]
\end{avm}

\section{V1-LR als DLR}\label{sec:v1lr}

\begin{avm}
  \[ \asort{lex-rule}
    loc|cat & \[
      head & \tuerkis{\[ \asort{verb}
        vform & fin \\
        \rot{initial} & \rot{$+$} \\
      \]}\\
      \gruen{subcat} & \<\[ loc|cat & \[
        head & \gruen{\[ \asort{verb}
          \blau{dsl} & \blau{\@1} \\
        \]}\\
        \gruen{subcat} & \gruen{\<\>} \\
      \]
      \]\> \\
    \] \\
    lex-dtr & \[
      \blau{loc} & \blau{\@1 \[ cat|head & \[ \asort{verb}
        vform & fin \\
        \orongsch{initial} & \orongsch{$-$} \\
      \]
    \]}
    \]
  \]
\end{avm}

\subsection{Spuren}

\subsubsection{Verbspur}

\begin{avm}
  \[ phon & \<\> \\
    loc  & \@{100} \[cat|head|dsl & \@{100}\] \\
  \]
\end{avm}

\subsubsection{Argumentspur}

\begin{avm}
	\[\asort{word}
		phon & \<\> \\
		loc  & \@{200} \\
		nonloc & \[slash & \<\@{200} \>\]
	\]
\end{avm}


\section{Aufbau des Satzes}

\subsection{Anwendung der V1-LR auf das lexikalische Verb}

\begin{avm}
  \[ \asort{lex-rule}
    phon & \<\rm\it liest\> \\
    loc|cat & \[
      head & \tuerkis{\[ \asort{verb}
        vform & fin \\
        \rot{initial} & \rot{$+$} \\
      \]}\\
      \gruen{subcat} & \<\[ loc|cat & \[
        head & \gruen{\[ \asort{verb}
          \blau{dsl} & \blau{\@1} \\
        \]}\\
        \gruen{subcat} & \gruen{\<\>} \\
      \]
      \]\> \\
    \] \\
    lex-dtr & \[ \asort{word}
    phon & \<\rm\it liest\> \\
    \blau{loc} & \blau{\@1 \[
      cat & \[
        head & \[ \asort{verb}
          vform & fin \\
          \orongsch{initial} & \orongsch{$-$} \\
        \]\\
        subcat & \<\[
          loc|cat & \[
            head & \[ \asort{noun}
              cas & nom \\
              num & sg \\
            \]\\
            subcat & \<\> \\
          \] \\
        \],
        \[
          loc|cat & \[
            head & \[ \asort{noun}
              cas & acc \\
            \]\\
            subcat & \<\> \\
          \] \\
        \]
        \> \\
      \] \\
    \]}
    \]
  \]
\end{avm}

\subsection{Verbindung des V1-Verbs mit der VP}\label{sec:top}


\begin{avm}
  \[ \asort{hd-arg-phrase}
    phon & \<\rm\it liest \_\_\_ \> \\
    cat & \[
      head & \tuerkis{\@3} \\
      subcat & \rot{\@4 \<\>} \\
    \] \\
    hd-dtr  & 
      \[ \asort{lex-rule}
        phon & \<\rm\it liest\> \\
        loc|cat & \[
          head & \tuerkis{\@3 \[ \asort{verb}
            vform & fin \\
            initial & $+$ \\
          \]}\\
          subcat & \rot{\@4}$\oplus$\<\orongsch{\@2}\> \\
        \] \\
        lex-dtr & \[ \asort{word}
        phon & \<\rm\it liest\> \\
        loc & \blau{\@1} 
        \]
    \] \\
    nhd-dtr &
      \orongsch{\@2\[
        loc|cat & \[
          head & \[ \asort{verb}
            \blau{dsl} & \blau{\@1
            \[
            cat & \[
              head & \[ \asort{verb}
                vform & fin \\
                initial & $-$ \\
              \]\\
              subcat & \< NP\Sub{nom}, NP\Sub{acc} \> \\
            \] \\
          \]}\\
          \]\\
          subcat & \<\> \\
        \]
      \]}
  \]
\end{avm}


\subsection{Aufbau der gespurten VP}\label{sec:vp}

\subsubsection{Verbindung der Verbspur mit \textit{Die Strudlhofstiege}}

\begin{avm}
  \[ \asort{hd-arg-phr}
    phon & \<\rm\it Die Strudlhofstiege \_ \> \\
    loc|cat & \[
      head & \rot{\@{105}} \\
      subcat & \braun{\@{102}}\\
    \]\\
    hd-dtr & \[
      phon & \<\> \\
      \tuerkis{loc}   & \tuerkis{\@{100} \[cat & \[
        head & \rot{\@{105} \[ dsl & \@{100} \]} \\
      subcat & \braun{\@{102}} $\oplus$ \< \@{101} \> \\
    \]
    \]} \\
    \]\\
    nhd-dtr & \tuerkis{\@{101}} \orongsch{\[\asort{word}
      phon & \<\rm\it Die Strudlhofstiege\> \\
      loc|cat & \[
        head & \[ \asort{noun}
          cas & acc\\
          num & sg \\
        \]\\
        subcat & \<\> \\
      \] \\
    \]}\\
  \]
\end{avm}

\subsubsection{Verbindung mit der Argumentspur}


\begin{avm}
  \[ \asort{hd-arg-phr}
    phon & \<\rm\it \_ Die Strudlhofstiege \_ \> \\
    loc|cat & \[
      head & \rot{\@{105}} \\ 
      subcat & \braun{\@{103}\ \<\>}\\
    \]\\
    nonloc & \[ slash & \<\@{200}\> \] \\
    hd-dtr &
    \[ \asort{hd-arg-phr}
      phon & \<\rm\it Die Strudlhofstiege \_ \> \\
      loc|cat & \[
        head & \rot{\@{105}} \\
        subcat & \braun{\@{102} \@{103}$\oplus$\<\gruen{\@{104}}\>}\\
      \]\\
      hd-dtr & \[
        phon & \<\> \\
        \tuerkis{loc}   & \tuerkis{\@{100} \[cat & \[
        head & \rot{\@{105} \[ dsl & \@{100} \]} \\
        subcat & \braun{\@{102}} $\oplus$ \< \@{101} \> \\
      \]
      \]} \\
      \]\\
      nhd-dtr & \tuerkis{\@{101}} \orongsch{\[\asort{word}
        phon & \<\rm\it Die Strudlhofstiege\> \\
        loc|cat & \[
          head & \[ \asort{noun}
            cas & acc\\
            num & sg \\
          \]\\
          subcat & \<\> \\
        \] \\
      \]}\\
    \]\\
    nhd-dtr & \gruen{\@{104}} \[ \asort{word}
      phon & \<\ \> \\
      loc & \@{200} \\
      nonloc & \[ slash & \<\@{200}\> \] \\
    \]
  \]
\end{avm}

\newpage

\subsection{Unifizierung der VP mit dem bewegten Verb}

\subsubsection{Abkürzung der NPs in der doppelt gespurten VP}


\begin{avm}
  \[ \asort{hd-arg-phr}
    phon & \<\rm\it \_ Die Strudlhofstiege \_ \> \\
    loc|cat & \[
      head & \rot{\@{105}} \\ 
      subcat & \braun{\@{103}\ \<\>}\\
    \]\\
    nonloc & \[ slash & \@{200} \] \\
    hd-dtr &
    \[ \asort{hd-arg-phr}
      phon & \<\rm\it Die Strudlhofstiege \_ \> \\
      loc|cat & \[
        head & \rot{\@{105}} \\
        subcat & \braun{\@{102} \@{103}$\oplus$\<\gruen{\@{104}}\>}\\
      \]\\
      hd-dtr & \[
        \tuerkis{loc}   & \tuerkis{\@{100} \[cat & \[
        head & \rot{\@{105} \[ dsl & \@{100} \]} \\
        subcat & \braun{\@{102}} $\oplus$ \< \@{101} \> \\
      \]
      \]} \\
      \]\\
      nhd-dtr & \tuerkis{\@{101} \textup{\textsc{NP\Sub{acc}}}}\\
    \]\\
    nhd-dtr & \gruen{\@{104} \[ loc & \@{200} \\
	    nonloc & \[ slash & \@{200}\] \\
    \] }
  \]
\end{avm}

\subsubsection{Umstellung | \textsc{NP\Sub{acc}}}

\begin{avm}
  \[ \asort{hd-arg-phr}
    phon & \<\rm\it \_ Die Strudlhofstiege \_ \> \\
    loc|cat & \[
      head & \rot{\@{105}} \\ 
      subcat & \braun{\@{103}\ \<\>}\\
    \]\\
    nonloc & \[ slash & \@{200}\] \\
    hd-dtr &
    \[ \asort{hd-arg-phr}
      phon & \<\rm\it Die Strudlhofstiege \_ \> \\
      loc|cat & \[
        head & \rot{\@{105}} \\
	subcat & \braun{\@{102} \@{103}$\oplus$\<\@{104} \[
		loc &\@{200}  \\
	\] \>}\\
      \]\\
      hd-dtr & \[
        \tuerkis{loc}   & \tuerkis{\@{100} \[cat & \[
        head & \rot{\@{105} \[ dsl & \@{100} \]} \\
        subcat & \braun{\@{102}} $\oplus$ \< \@{101} \textup{\textsc{NP\Sub{acc}}} \> \\
      \]
      \]} \\
      \]\\
    \]\\
  \]
\end{avm}

\subsubsection{Umstellung | Information der Spur}


\begin{avm}
  \[ \asort{hd-arg-phr}
    phon & \<\rm\it \_ Die Strudlhofstiege \_ \> \\
    loc|cat & \[
      head & \rot{\@{105}} \\ 
      subcat & \braun{\<\>}\\
    \]\\
    nonloc & \[ slash & \@{200}\] \\
    hd-dtr &
    \[ \asort{hd-arg-phr}
      phon & \<\rm\it Die Strudlhofstiege \_ \> \\
      loc|cat & \[
        head & \rot{\@{105}} \\
      \]\\
      hd-dtr & \[
        \tuerkis{loc}   & \tuerkis{\@{100} \[cat & \[
        head & \rot{\@{105} \[ dsl & \@{100} \]} \\
	subcat &  \< \@{104} \[ loc & \@{200}\],
	\@{101} \textup{\textsc{NP\Sub{acc}}} \> \\
      \]
      \]} \\
      \]\\
    \]\\
  \]
\end{avm}

\subsubsection{Reduktion der gespurten VP aufs Wesentliche}

\begin{avm}
  \[ \asort{hd-arg-phr}
    phon & \<\rm\it \_ Die Strudlhofstiege \_ \> \\
    loc|cat & \[
      head & \rot{\@{105} \[ dsl & \@{100} \]} \\ 
      subcat & \braun{\<\>}\\
    \]\\
    nonloc & \[ slash & \<\@{200}\> \] \\
    hd-dtr &
    \[ hd-dtr & \[
        \tuerkis{loc}   & \tuerkis{\@{100} \[cat & \[
        head & \rot{\@{105}} \\
	subcat & \< \@{104} \[
		loc & \@{200}
	\], \@{101} \textup{\textsc{NP\Sub{acc}}} \> \\
      \]
      \]} \\
      \]\\
    \]\\
  \]
\end{avm}

\Zeile

Wieder gilt, dass diese Struktur mit dem \textsc{subcat}-Eintrag des bewegten Verbs aus Abschnitt~\ref{sec:top} unifiziert.

\Zeile

\begin{avm}
  \[
    loc|cat & \[
      head & \[ \asort{verb}
        dsl & \@1
        \[
        cat & \[
          head & \[ \asort{verb}
            vform & fin \\
            initial & $-$ \\
          \]\\
          subcat & \< NP\Sub{nom}, NP\Sub{acc} \> \\
        \] \\
      \]\\
      \]\\
      subcat & \<\> \\
    \]
  \]
\end{avm}

\Zeile

Beachten Sie, dass die folgenden beiden Strukturen unifizierbar sind, solange wir (wie hier) über \mybox{200} noch rein gar nichts gesagt haben.
Einerseits die Information über die Spur, die auf der \textsc{subcat} der Spur steht: 

\Zeile

\begin{avm}
  \@{104}\ \[loc & \@{200}\]
\end{avm}

\Zeile

Und die Anforderung des bewegten Verbs, dass es eine \textsc{NP\Sub{nom}} als erstes Element auf der \textsc{subcat} hat:

\Zeile

\begin{avm}
  \[loc & \[
    cat & \[ head & \[ \asort{noun}
    cas & nom \\
    num & sg \\
  \] \\
    subcat & \<\> \\
  \]
  \]
  \]
\end{avm}

\Zeile

Wir erhalten ganz einfach:

\Zeile

\begin{avm}
  \@{104}\ \[loc & \@{200} \[
    cat & \[ head & \[ \asort{noun}
    cas & nom \\
    num & sg \\
  \] \\
    subcat & \<\> \\
  \]
  \]
  \]
\end{avm}

\subsubsection{Kombination der V1-VP}

\begin{avm}
  \[ \asort{hd-arg-phrase}
    phon & \<\rm\it Liest \_ Die Strudlhofstiege \_\> \\
    loc|cat & \[
      head & \tuerkis{\@3} \\
      subcat & \rot{\@4 \<\>} \\
    \] \\
    nonloc & \[ slash & \@{200}\] \\
    hd-dtr  & 
      \[ \asort{lex-rule}
        phon & \<\rm\it liest\> \\
        loc|cat & \[
          head & \tuerkis{\@3 \[ \asort{verb}
            vform & fin \\
            initial & $+$ \\
          \]}\\
          subcat & \rot{\@4}$\oplus$\<\orongsch{\@2}\> \\
        \] \\
        lex-dtr & \[ \asort{word}
        phon & \<\rm\it liest\> \\
        loc & \blau{\@1} 
        \]
    \] \\
    nhd-dtr &
      \orongsch{\@2\[
        phon & \<\rm\it \_ Die Strudlhofstiege \_\> \\
        loc|cat & \[
          head & \[ \asort{verb}
            \blau{dsl} & \blau{\@1
            \[
            cat & \[
              head & \[ \asort{verb}
                vform & fin \\
                initial & $-$ \\
              \]\\
	      subcat & \< \@{104} \[
		      loc & \@{200} \[
			      head & \[
				      case & nom \\
				      num & sg \\
			      \] \\
		      \] \\
	      \], \@{101} NP\Sub{acc} \> \\
            \] \\
          \]}\\
          \]\\
          subcat & \<\> \\
        \]
      \]}
  \]
\end{avm}

\subsubsection{Bau der Head-Filler Phrase}

Das wäre dann die Basis-Hausaufgabe.
Hier ist der Anfang:

\Zeile

\begin{avm}
  \[ \asort{hd-filler-phr}
    phon & \<\rm\it Kiki liest \_ Die Strudlhofstiege \_ \> \\
%
    loc|cat & \[
      head & \@{3} \\
      subcat & \@{4}\<\> \\ 
    \] \\
    nonloc|slash & \<\> \\
%
  filler-dtr & \[ \asort{word}
    phon & \<\rm\it Kiki \> \\
    loc \@{200} cat & \[
      head & \[ \asort{noun}
        cas & nom \\
        num & sg \\
      \]\\
      subcat & \<\> \\
    \] \\
    nonloc|slash & \<\> \\
  \] \\
%
    hd-dtr & \[ \asort{hd-arg-phrase}
    phon & \<\rm\it Liest \_ Die Strudlhofstiege \_\> \\
    loc|cat & \[
      head & \tuerkis{\@3} \\
      subcat & \rot{\@4 \<\>} \\
    \] \\
    nonloc & \[ slash & \@{200}\] \\
    hd-dtr  & 
      \[ \asort{lex-rule}
        phon & \<\rm\it liest\> \\
        loc|cat & \[
          head & \tuerkis{\@3 \[ \asort{verb}
            vform & fin \\
            initial & $+$ \\
          \]}\\
          subcat & \rot{\@4}$\oplus$\<\orongsch{\@2}\> \\
        \] \\
        lex-dtr & \[ \asort{word}
        phon & \<\rm\it liest\> \\
        loc & \blau{\@1} 
        \]
    \] \\
    nhd-dtr &
      \orongsch{\@2\[
        phon & \<\rm\it \_ Die Strudlhofstiege \_\> \\
        loc|cat & \[
          head & \[ \asort{verb}
            \blau{dsl} & \blau{\@1
            \[
            cat & \[
              head & \[ \asort{verb}
                vform & fin \\
                initial & $-$ \\
              \]\\
	      subcat & \< \@{104} \[
		      loc & \@{200} \[
			      head & \[
				      case & nom \\
				      num & sg \\
			      \] \\
		      \] \\
	      \], \@{101} NP\Sub{acc} \> \\
            \] \\
          \]}\\
          \]\\
          subcat & \<\> \\
        \]
      \]}
  \]
  \]
\end{avm}


\section{Zusatzaufgaben}

\begin{enumerate}
  \item Überlegen Sie, wie man einen einfachen Relativsatz modellieren könnte.
    Es geht um, sowas wie das Fettgedruckte in (\ref{ex:rel0}).
    \begin{enumerate}
      \item Was wird in einem Relativsatz wohin bewegt?
        Machen Sie sich also zuerst Gedanken über die deskriptiven Unterschiede zwischen VL-Sätzen, unabhängigen V2-Sätzen und Relativsätzen.
      \item Für die relevante Bewegung im Relativsatz verwendet man \textsc{nonloc|rel} statt \textsc{nonloc|slash}.
        Überlegen Sie, was darauf abgelegt werden muss, und überlegen Sie, wie der Filler-Mechanismus dazu aussehen muss.
        Was ist gleich wie, was ist anders als bei \textsc{slash}?
    \end{enumerate}
    \Halbzeile
  \item Was ist das Problem an der Modellierung von komplexeren Relativsätzen wie in (\ref{ex:rel})?
    Überlegen Sie insbesondere, was die Probleme mit der Aufteilung der Merkmalstruktur in lokale Information inkl.\ \textsc{head} und nichtlokale Information sein könnten.
\end{enumerate}

\Zeile

\begin{exe}
  \ex\label{ex:rel0} das Buch, \textbf{das ich lese}
  \ex\label{ex:rel} das Buch, \textbf{dessen Inhalt ich kenne}
\end{exe}

\end{document}
