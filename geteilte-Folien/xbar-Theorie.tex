\subsection{\xbar-Syntax}

\subsubsection{Nominalphrasen}
\label{sec-psg-np}

\frame{
\frametitle{Nominalphrasen}

\begin{itemize}
\item Bisher NPen immer Det + N, Nominalphrasen können aber wesentlich komplexer sein:
\eal
\ex ein Buch
\ex ein Buch, das wir kennen
\ex ein Buch aus Japan
\ex ein interessantes Buch
\ex ein Buch aus Japan, das wir kennen
\ex ein interessantes Buch aus Japan
\ex ein interessantes Buch, das wir kennen
\ex ein interessantes Buch aus Japan, das wir kennen
\zl

Zusätzliches Material in (\mex{0}) sind Adjunkte.

\end{itemize}

}

\frame{
\frametitle{Adjektive in NPen}

\begin{itemize}
\item Vorschlag:
\eal
\ex NP $\to$ Det N
\ex NP $\to$ Det A N
\zl
\pause
\item Was ist mit (\mex{1})?
\ea
alle weiteren schlagkräftigen Argumente
\z
\pause
\item Für die Analyse von (\mex{0}) müsste man eine Regel wie (\mex{1}) haben:
\ea 
NP $\to$ Det A A N
\z
\pause
\item Wir wollen keine Höchtszahl für Adjektive in NPen angeben: 
\ea 
NP $\to$ Det A* N
\z
\end{itemize}

}

\frame{
\frametitle{Adjektive in NPen}

\begin{itemize}

\item Problem: Adjektiv und Nomen bilden bei Annahme von (\mex{1}) keine Konstituente.
\ea 
NP $\to$ Det A* N
\z
Konstituententests legen aber Konstituentenstatus von A + N nahe:
\ea
alle [[großen Seeelefanten] und [grauen Eichhörnchen]]
\z

\end{itemize}

}

\frame{
\frametitle{Adjektiv + Nomen als Konstituente}

\begin{itemize}
\item Besser geeignete Regeln:
\eal
\ex NP $\to$ Det \nbar
\ex \nbar $\to$ A \nbar
\ex \nbar $\to$ N
\zl


\hfill%
\scalebox{.65}{%
\begin{forest}
sm edges
[NP
   [Det [ein] ]
   [\nbar
      [N [Eichhörnchen] ] ] ]
\end{forest}}
\hfill
\scalebox{.65}{%
\begin{forest}
sm edges
[NP
   [Det [ein] ]
   [\nbar
      [A [graues] ]
      [\nbar
        [N [Eichhörnchen] ] ] ] ]
\end{forest}}
%
\hfill
\scalebox{.65}{%
\begin{forest}
sm edges
[NP
  [Det [eine] ]
    [\nbar
    [A [großes] ]
       [\nbar
       [A [graues] ]
         [\nbar
         [N [Eichhörnchen] ] ] ] ] ]
\end{forest}}
\hfill\mbox{}
%
\end{itemize}

}





% Das Adjektiv \emph{klug} schränkt die Menge der Bezugselemente der Nominalgruppe ein. Nimmt man ein
% weiteres Adjektiv wie \emph{glücklich} dazu, dann bezieht man sich nur auf die Frauen, die sowohl glücklich
% als auch klug sind. Solche Nominalphrasen können \zb in Kontexten wie dem folgenden verwendet
% werden:
% \ea
% \label{Beispiel-Iteration-Adjektive}
% A: Alle klugen Frauen sind unglücklich.\\
% B: Nein, ich kenne eine glückliche kluge Frau.
% \z
% Man kann sich nun überlegen, dass dieses schöne Gespräch mit Sätzen wie \emph{Aber alle glücklichen
%   klugen Frauen sind schön} und einer entsprechenden Antwort weitergehen kann. Die Möglichkeit,
% Nominalphrasen wie \emph{eine glückliche kluge Frau} weitere Adjektive hinzuzufügen, ist im
% Regelsystem in (\mex{-1}) angelegt.

% Wir haben jetzt eine wunderschöne kleine Grammatik entwickelt, die Nominalphrasen mit
% Adjektivmodifikatoren analysieren kann. Dabei wird der Kombination von Adjektiv und Nomen
% Konstituentenstatus zugesprochen. Der Leser wird sich jetzt vielleicht fragen, ob man nicht genauso
% gut die Kombination aus Determinator und Adjektiv als Konstituente behandeln könnte, denn es gibt
% auch Nominalgruppen der folgenden Art:
% \ea
% diese schlauen und diese neugierigen Frauen
% \z
% Hier liegt jedoch noch eine andere Struktur vor: Koordiniert sind zwei vollständige Nominalphrasen,
% wobei im ersten Konjunkt ein Teil gelöscht wurde:% \footnote{
% %   Außerdem gibt es bei entsprechendem Kontext natürlich noch die Lesart, in der sich \emph{diese schlauen} nicht auf Frauen
% %   sondern \zb auf Männer bezieht:
% % \ea
% % Hier stehen mehrere Gruppen von Männern
% \ea
% diese schlauen \st{Frauen} und diese neugierigen Frauen
% \z
% Man findet ähnliche Phänomene im Satzbereich und sogar bei Wortteilen:
% \eal
% \ex dass Peter dem Mann das Buch \st{gibt} und Maria der Frau die Schallplatte gibt
% \ex be- und entladen
% \zl
% % Dass in (\mex{-1}) wirklich keine normale symmetrische Koordination vorliegt, sieht man, wenn man
% % (\mex{-1}) mit (\mex{1}) vergleicht:
% % \ea
% % diese schlauen Frauen und klugen Männer
% % \z
% % Mit (\mex{0}) verweist man auf eine Gruppe, die aus schlauen Frauen und klugen Männern besteht,
% % wohingegen man mit (\mex{-2}) auf zwei Gruppen verweist, nämlich

\frame{
\frametitle{Andere Adjunkte}


\begin{itemize}
\item Andere Adjunkte analog:
\eal
\ex \nbar $\to$ \nbar PP
\ex \nbar $\to$ \nbar Relativsatz
\zl
\pause
\ex Mit den bisher aufgeführten Regeln können wir alle bisher genannten
Determinator-Adjunkt-Nomen-Kombinationen analysieren.

\end{itemize}

}

\frame{
\frametitle{Komplemente}


\begin{itemize}
\item Bisher besteht \nbar nur aus einem Nomen,\\
      aber einige Nomina erlauben neben Adjunkten auch Argumente:
\eal
\ex der Vater von Peter
\ex das Bild vom Gleimtunnel
\ex das Kommen des Installateurs
\zl

\pause
\item Deshalb:
\ea
\nbar $\to$ N PP
\z

\end{itemize}

}

\frame{
\frametitle{Komplemente (und Adjunkte)}

\hfill

\centerfit{%
\begin{forest}
sm edges
[NP
 [Det [das] ]
 [\nbar
   [N [Bild] ]
   [PP [vom Gleimtunnel,roof ] ] ] ]
\end{forest}%
\hspace{2em}%
\begin{forest}
sm edges
[NP
  [Det [das] ]
  [\nbar
    [\nbar
      [N [Bild] ]
      [PP [vom Gleimtunnel,roof ] ] ] 
    [PP [im Gropiusbau,roof ] ] ] ]
\end{forest}}


}

\frame[shrink=25]{
\frametitle{Fehlende Nomina}

\savespace
\begin{itemize}
\item Nomen fehlt, aber Adjunkte sind vorhanden:
\eal
\ex ein interessantes \_
\ex ein interessantes \_ aus Hamburg
\ex ein interessantes \_, das alle kennen
\zl
\pause
\item Nomen fehlt, aber Komplement des Nomens ist vorhanden:
\eal
\ex (Nein, nicht der Vater von Klaus),
    der \_ von Peter war gemeint.
\ex (Nein, nicht das Bild von der Stadtautobahn),
    das \_ vom Gleimtunnel war beeindruckend.
\ex (Nein, nicht das Kommen des Tischlers),
    das \_ des Installateurs ist wichtig.
\zl
\pause
\item PSG: \alert{Epsilonproduktion}
\pause
\item Notationsvarianten:
\eal
\ex N $\to$
\ex N $\to$ $\epsilon$
\zl 

\pause
\item Regeln in (\mex{0}) = leeren Schachteln, die aber dieselbe
Beschriftung tragen, wie die Schachteln normaler Nomina. 

\end{itemize}
}

\frame{
\frametitle{Analysen mit leerem Nomen}

\hfill
\begin{forest}
sm edges
[NP
  [Det [ein] ]
  [\nbar
    [A [interessantes] ]
    [\nbar
      [N [\trace ] ] ] ] ]
\end{forest}
\hfill
\begin{forest}
sm edges
[NP
  [Det [das] ]
  [\nbar
    [N [\trace] ]
    [PP [vom Gleimtunnel,roof] ] ] ]
\end{forest}
\hfill%
\mbox{}

}


\frame{
\frametitle{Fehlende Determinatoren}

\begin{itemize}
\item Auch Determinatoren können weggelassen werden.

Plural:
\eal
\ex Bücher
\ex Bücher, die wir kennen
\ex interessante Bücher
\ex interessante Bücher, die wir kennen
\zl

\pause
\item Bei Stoffnomen auch im Singular:
\eal
\ex Getreide
\ex Getreide, das gerade gemahlen wurde
\ex frisches Getreide
\ex frisches Getreide, das gerade gemahlen wurde
\zl

\end{itemize}

}


\frame{
\frametitle{Fehlende Determinatoren}


\centering
\begin{forest}
sm edges
[NP
  [Det [\trace] ]
  [\nbar
    [N [Bücher] ] ] ]
\end{forest}

}

\frame{
\frametitle{Fehlende Determinatoren und fehlende Nomina}

Determinator und Nomen können auch gleichzeitig weggelassen werden:
\eal
\ex Ich lese interessante.
\ex Dort drüben steht frisches, das gerade gemahlen wurde.
\zl

\centerline{%
\scalebox{.9}{%
\begin{forest}
sm edges
[NP
  [Det [\trace] ]
  [\nbar
    [A [interessante] ]
    [\nbar
      [N [\trace] ] ] ] ]
\end{forest}}
}
}

% % Mit diesen wenigen Phrasenstrukturregeln sind die wesentlichen Aspekte der Nominalsyntax
% % abgedeckt. Zwei Probleme werden dem aufmerksamen Leser nicht entgangen sein: Erstens muss man, wenn
% % eine Nominalstruktur kein Adjektiv enthält, ausschließen, dass sowohl der Determinator als auch das
% % Nomen leer sind, denn ansonsten würde man eine leere Nominalphrase ableiten. Das kann man mit
% % zusätzlichen Merkmalen sicherstellen \citep{Netter98a}.\NOTE{genaue Quellenangabe} Zweitens darf das
% % Nomen nicht entfallen, wenn kein Adjektiv in der Nominalgruppe gibt:
% % \eal
% % \ex[]{
% % Wir helfen den Männern.
% % }
% % \ex[]{
% % Wir helfen den schlauen.
% % }
% % \ex[*]{
% % Wir helfen den.
% % }
% % \ex[]{
% % Wir helfen denen.
% % }
% % \ex[]{
% % Wir helfen denen mit Hut.
% % }
% % \ex[]{
% % Wir helfen denen, die wir kennen.
% % }
% % \zl
% % In den Fällen ohne Adjektiv muss das Demonstrativpronomen verwendet werden.
% %
% %
% % Da man in elliptischen Kontexten beim
% % Auspacken einer ähnlich beschrifteten Schachtel bereits etwas gefunden hat, muss man die leere
% % Schachtel nicht mehr auspacken und so fällt es nicht auf, dass sie nichts enthält.


% %%%%%%%%%%%%%%%%%%%%%%%%%%%%%%%%%%%%%%%%%%%%%%%%%%%%%%%%%%%%%%%%%%%%%%%%%%%%%%%%%%%%%%%%%%%%%%%%%%

\subsubsection{Adjektivphrasen}

\frame[shrink]{
\frametitle{Adjektivphrasen}

\begin{itemize}
\item Bisher nur einfache Adjektive wie \emph{interessant}.
\pause
\item Mitunter sind Adjektivphrasen aber sehr komplex:
\eal
\ex der [seiner Frau treue] Mann
\ex der [auf seine Tochter stolze] Mann
\ex der [seine Frau liebende] Mann
\ex der [von seiner Frau geliebte] Mann
\zl
\pause
\item \dash, Regel für attributive Adjektive muss angepasst werden:
\ea
\nbar $\to$ AP \nbar
\z
\pause
\item
Regeln für AP:
\eal
\ex AP $\to$ NP A
\ex AP $\to$ PP A
\ex AP $\to$ A
\zl

\end{itemize}

}

\if 0

In den bisher ausgearbeiteten Regeln gibt es zwei Unschönheiten. Das sind die Regeln für Adjektive
bzw.\ Nomina ohne Komplemente in (\mex{0}c) bzw.\ (\ref{NP-Regeln-Nbar-N}) --  hier als (\mex{1}) wiederholt:
\ea
\nbar $\to$ N
\z
Werden diese Regeln angewendet, ergeben sich Teilbäume mit unärer Verzweigung, \dash mit einer Mutter,
die nur eine Tochter hat. Für ein Beispiel siehe Abbildung~\ref{Abbildung-Adjektive-in-NP}. Wenn wir
bei unserem Gleichnis mit den Schachteln bleiben, heißt das, dass es eine Schachtel gibt, die eine
Schachtel enthält, in der dann der eigentlich relevante Inhalt steckt. 

Im Prinzip hindert uns aber nichts, den Inhalt gleich in die größere Schachtel zu tun. Statt der
Regeln in (\mex{1}) verwenden wir einfach die Regeln in (\mex{2}):
\eal
\ex A $\to$ kluge
\ex N $\to$ Mann
\zl
\eal
\label{Lexikon-Projektion}
\ex AP $\to$ kluge
\ex \nbar $\to$ Mann
\zl
Mit (\mex{0}a) wird ausgedrückt, dass \emph{kluge} dieselben Eigenschaften wie vollständige
Adjektivphrasen hat, insbesondere kann es nicht mehr mit einem Komplement kombiniert werden. Das ist
parallel zur Kategoriesierung des Pronomens \emph{er} als NP in den Grammatiken
(\ref{bsp-grammatik-psg}) und (\ref{psg-binaer}).


Die Einordnung von Nomina, die kein Komplement verlangen, als \nbar hat außerdem auch den Vorteil,
dass man nicht erklären muss, warum es neben (\mex{1}a) auch noch die Analyse (\mex{1}b) geben soll, obwohl es
keinen Bedeutungsunterschied gibt.
\eal
\ex {}[\sub{NP} einige [\sub{\nbar} kluge [\sub{\nbar} [\sub{\nbar} Frauen ] und [\sub{\nbar} Männer
]]]]
\ex {}[\sub{NP} einige [\sub{\nbar} kluge [\sub{\nbar} [\sub{N} [\sub{N} Frauen ] und [\sub{N} Männer
]]]]]
\zl
In (\mex{0}a) sind zwei Nomina der Kategorie \nbar koordinativ verknüpft worden. Das Ergebnis einer
Koordination zweier Konstituenten gleicher syntaktischer Kategorie ist immer einer neue Konstituente
derselben syntaktischen Kategorie, in (\mex{0}a) also ebenfalls eine \nbar. Diese wird dann mit dem
Adjektiv und dem Determinator kombiniert.
In (\mex{0}b)) wurden die Nomina kombiniert. Das Ergebnis ist wieder eine Konstituente, die dieselbe
Kategorie hat, wie ihre Teile, also ein N. Dieses N wird zur \nbar, die dann mit dem Adjektiv
verbunden wird. Wenn man Nomina, die kein Komplement verlangen, nicht als N sondern als \nbar
kategorisiert, ergibt sich das beschriebene Problem mit sogenannten unechten Mehrdeutigkeiten
nicht. 

\fi

\subsubsection{Präpositionalphrasen}

\frame[shrink=5]{
\frametitle{Präpositionalphrasen}

\savespace
\begin{itemize}
\item PP-Syntax ist relativ einfach. Erster Vorschlag:
\ea
PP $\to$ P NP
\z
\pause
\item Allerdings können PPen durch Maßangaben oder andere Angaben,\\
      die den Bedeutungsbeitrag der Präposition konkretisieren, erweitert werden:
\eal
\ex {}[[Einen Schritt] vor dem Abgrund] blieb er stehen.\label{Beispiel-Schritt-vor-dem-Abgrund}
\ex {}[[Kurz] nach dem Start] fiel die Klimaanlage aus.
\ex {}[[Schräg] hinter der Scheune] ist ein Weiher.
\ex {}[[Mitten] im Urwald] stießen die Forscher auf einen alten Tempel.
\zl

% Man könnte jetzt für die Analyse von (\mex{0}a,b) eine Regel wie in (\mex{1}) vorschlagen:
% \eal
% \ex PP $\to$ NP PP
% \ex PP $\to$ AP PP
% \zl
% Die Regeln kombinieren eine PP mit einer Maßangabe. Das Ergebnis ist wieder eine PP. Mit den Regeln
% könnte man zwar die Präpositionalphrasen in (\mex{-1}a,b) analysieren, aber leider auch die in
% (\mex{1}):
% \eal
% \ex[*]{
% einen Schritt kurz vor dem Abgrund
% }
% \ex[*]{
% kurz einen Schritt vor dem Abgrund
% }
% \zl
% In (\mex{0}) wurden jeweils beide Regeln aus (\mex{-1}) angewendet.

% Durch Umformulierung der bisherigen Regeln kann man diesen Nebeneffekt vermeiden:
\eal
\ex PP $\to$ NP \pbar
\ex PP $\to$ AP \pbar
\ex PP $\to$ \pbar\label{Regel-PP-P}
\ex \pbar $\to$ P NP
\zl
\end{itemize}

}

\frame{
\frametitle{Präpositionalphrasen}

\hfill
\hfill
\begin{forest}
sm edges
[PP
  [\pbar
    [P [vor] ]
    [NP [dem Abgrund,roof] ] ] ]
\end{forest}
\hfill
\begin{forest}
sm edges
[PP
  [AP [kurz,roof] ]
  [\pbar
    [P [vor] ]
    [NP [dem Abgrund,roof] ] ] ]
\end{forest}
\hfill
\mbox{}

}



\subsubsection{Die \xbar-Theorie}
\label{sec-xbar}

\frame{
\frametitle{Generalisierungen über Regeln}

\begin{itemize}
\item Kopf + Komplement = Zwischenstufe:
\eal
\ex \nbar $\to$ N PP
\ex \pbar $\to$ P NP
\zl
\pause
\item Zwischenstufe + weiter Konstituente = Maximalprojektion
\eal
\ex NP $\to$ Det \nbar
\ex PP $\to$ NP \pbar
\zl
\pause
\item parallele Strukturen auch für AP und VP im Englischen
\end{itemize}

}


\frame{
\frametitle{Adjektivphrasen im Englischen}

\eal
\ex They are proud.
\ex They are very proud.
\ex They are proud of their child.
\ex They are very proud of their child.
\zl

\pause

\eal
\ex AP $\to$ \abar
\ex AP $\to$ Adv \abar
\ex \abar $\to$ A PP
\ex \abar $\to$ A
\zl

}


\frame{
\frametitle{Adjektivphrasen im Englischen}

\eal
\ex AP $\to$ \abar
\ex AP $\to$ AdvP \abar
\ex \abar $\to$ A PP
\ex \abar $\to$ A
\zl


\hfill
\begin{forest}
sm edges
[AP
  [\abar
    [A [proud] ] ] ]
\end{forest}
\hfill
\begin{forest}
sm edges
[AP
  [AdvP [very] ]
  [\abar
    [A [proud] ] ] ]
\end{forest}
\hfill
\begin{forest}
sm edges
[AP
  [\abar
    [A [proud] ]
    [PP [of their child,roof] ] ] ]
\end{forest}
\hfill
\begin{forest}
sm edges
[AP
  [AdvP [very] ]
  [\abar
    [A [proud] ]
    [PP [of their child,roof] ] ] ]
\end{forest}
\hfill
\mbox{}
\hfill
\mbox{}

}

\frame{
\frametitle{Weitere Abstraktion}

\begin{itemize}
\item Haben gesehen, wie man über Kasus- und Genuswerte u.\,ä. abstrahieren kann (Variablen in
  Regelschemata).

\ea
NP({3},{Num},{Kas}) $\to$ D(Gen,{Num},{Kas}), N(Gen,{Num},{Kas})
\z

\pause
\item Genauso kann man über Wortart abstrahieren.\\
      Statt AP, NP, PP, VP schreibt man XP.
\pause
\item Satt (\mex{1}), schreibt man (\mex{2}):
\eal
\ex PP $\to$ \pbar
\ex AP $\to$ \abar
\zl
\ea
XP $\to$ \xbar
\z
\end{itemize}


}


\frame{
\frametitle{\xbar-Theorie: Annahmen (II)}

Phrasen sind mindestens dreistöckig:
\begin{itemize}
\item X$^0$ = Kopf
\item X$'$ = Zwischenebene (= \xbar, sprich X-Bar, X-Strich; $\to$ Name des Schemas) 
\item XP = oberster Knoten (=~X$''$ = $\overline{\overline{\mbox{X}}}$), auch Maximalprojektion genannt
\end{itemize}
%Neuere Analysen $\to$ teilweise Verzicht auf nichtverzweigende X$'$-Knoten
\nocite{Muysken82a}


}

\frame[shrink]{
\frametitle{Minimaler und maximaler Ausbau von Phrasen}

\bigskip

\small\hfill
\begin{forest}
sm edges
[XP
  [\xbar [X] ] ]
\end{forest}
\hfill
\begin{forest}
%where n children=0{}{},
%sm edges
%for tree={parent anchor=south, child anchor=north,align=center,base=bottom}
[XP
  [specifier]
  [\xbar
    [adjunct]
    [\xbar
      [complement] [X] ] ] ]
\end{forest}
\hfill\mbox{}


\begin{itemize}
\item Adjunkte sind optional\\
$\to$ muss nicht unbedingt ein X$'$ mit Adjunkttochter geben.
\pause
\item Für manche Kategorien gibt es keinen Spezifikator, oder er ist optional (\zb A).\\
%(Zusätzliche Regel nötig $\overline{\overline{\mbox{X}}} \rightarrow \xbar$)
\pause
\item zusätzlich mitunter: Adjunkte an XP und Kopfadjunkte an X. 
\iftoggle{gb-intro}{
(dazu \hyperlink{inkorporation}{später})
}
\end{itemize}

}

\frame{
\frametitle{\xbar-Theorie: Regeln nach \citew{Jackendoff77a}}\nocite{KP90a}\nocite{Pullum85a}



\oneline{\(
\begin{array}{@{}l@{\hspace{1cm}}l@{\hspace{1cm}}l}
\xbar\mbox{-Regel} & \mbox{mit Kategorien} & \mbox{Beispiel}\\[2mm]
\overline{\overline{\mbox{X}}} \rightarrow \overline{\overline{\mbox{Spezifikator}}}~~\xbar &
                                                                                              \overline{\overline{\mbox{N}}} \rightarrow \overline{\overline{\mbox{DET}}}~~\overline{\mbox{N}} & \mbox{das [Bild vom Gleimtunnel]} \\
\xbar \rightarrow \xbar~~\overline{\overline{\mbox{Adjunkt}}}             & \overline{\mbox{N}}
                                                                            \rightarrow
                                                                            \overline{\mbox{N}}~~\overline{\overline{\mbox{REL\_SATZ}}}
                                           & \mbox{[Bild vom Gleimtunnel] [das alle kennen]}\\
\xbar \rightarrow \overline{\overline{\mbox{Adjunkt}}}~~\xbar             & \overline{\mbox{N}}
                                                                            \rightarrow
                                                                            \overline{\overline{\mbox{ADJ}}}~~\overline{\mbox{N}}
                                           & \mbox{schöne [Bild vom Gleimtunnel]}\\
\xbar \rightarrow \mbox{X}~~\overline{\overline{\mbox{Komplement}}}*               &
                                                                                     \overline{\mbox{N}} \rightarrow \mbox{N}~~\overline{\overline{\mbox{P}}} & \mbox{Bild [vom Gleimtunnel]}\\\\
\end{array}
\)}

X steht für beliebige Kategorie, X ist Kopf,\\
`*' steht für beliebig viele Wiederholungen

\medskip
X kann links oder rechts in Regeln stehen

}


\if 0
\begin{figure}[htb]
\hfill
\begin{tabular}[b]{@{}ccccc@{}}
\multicolumn{2}{c}{\node{np}{NP}}\\[3ex]
%
\node{detp}{DetP}    & \multicolumn{2}{c}{\node{nbar}{\nbar}}\\[3ex]
%
\node{detbar}{\detbar} & \node{ap}{AP}     & \multicolumn{2}{c}{\node{nbar2}{\nbar}}\\[3ex]
%
\node{det}{Det}     & \node{abar}{\abar}  & \node{n}{N}    & \multicolumn{2}{c}{\node{pp}{PP}}\\[3ex]
%
                    & \node{A}{A}         &                & \multicolumn{2}{c}{\node{pbar}{\pbar}}\\[3ex]
%
                    &                     &      & \node{p}{P}   & \node{npmaria}{NP}\\[3ex]
%
        &        &      &     & \node{nbar3}{\nbar}\\[3ex]
%
        &        &      &     & \node{n2}{N}\\[3ex]
%
\node{das}{das}     & \node{schoene}{schöne} & \node{Bild}{Bild} & \node{von}{von} & \node{maria}{Paris}\\
\end{tabular}
\nodeconnect{np}{detp}\nodeconnect{np}{nbar}%
\nodeconnect{detp}{detbar}\nodeconnect{detbar}{det}\nodeconnect{det}{das}%
\nodeconnect{nbar}{ap}\nodeconnect{nbar}{nbar2}%
\nodeconnect{nbar2}{n}\nodeconnect{nbar2}{pp}%
\nodeconnect{pp}{pbar}\nodeconnect{pbar}{p}%
\nodeconnect{pbar}{npmaria}\nodeconnect{p}{von}%
\nodeconnect{npmaria}{nbar3}\nodeconnect{nbar3}{n2}\nodeconnect{n2}{maria}%
\nodeconnect{ap}{abar}\nodeconnect{abar}{A}\nodeconnect{A}{schoene}%
\nodeconnect{n}{Bild}
%
%
\hfill
\begin{tabular}[b]{@{}cc@{}}
\multicolumn{2}{c}{\node{np-sim}{NP}}\\[3ex]
%
\node{detp-sim}{DetP}    & \node{nbar-sim}{\nbar}\\[3ex]
%
\node{detbar-sim}{\detbar} & \node{n-sim}{N}\\[3ex] 
%
\node{det-sim}{Det}\\[3ex]
%
\node{das-sim}{das}     & \node{Bild-sim}{Bild}\\
\end{tabular}
\nodeconnect{np-sim}{detp-sim}\nodeconnect{np-sim}{nbar-sim}%
\nodeconnect{detp-sim}{detbar-sim}\nodeconnect{detbar-sim}{det-sim}\nodeconnect{det-sim}{das-sim}%
\nodeconnect{nbar-sim}{n-sim}\nodeconnect{n-sim}{Bild-sim}%
\hfill\mbox{}
\caption{\label{Abb-das-schoene-Bild-von-Paris}\xbar"=Analyse von \emph{das schöne Bild von Paris}
  und \emph{das Bild}}
\end{figure}
In der abgebildeten Analyse geht man davon aus, dass wirklich alle Nicht"=Köpfe Phrasen sind, man
nimmt also auch an, dass es eine Determinatorphrase gibt, obwohl die Determinatoren nicht mit
anderen Elementen kombiniert werden. Abbildung~\vref{Abb-GB-Min-Max} zeigt den minimalen und
maximalen Ausbau von Phrasen.
\begin{figure}[htb]
\centerline{%
\begin{tabular}[t]{c}
\node{xp}{XP}\\[5ex]
\node{xs}{X$'$}\\[5ex]
\node{x}{X}\\
\end{tabular}%
\nodeconnect{xp}{xs}\nodeconnect{xs}{x}\hspace{10ex}%
\begin{tabular}[t]{cccc}
\multicolumn{2}{c}{\hspace{18mm}\node{xp2}{XP}}\\[5ex]
\node{spec}{Spezifikator} & \multicolumn{2}{c}{\node{xs2}{X$'$}}\\[5ex]
                          & \node{adj}{Adjunkt} & \multicolumn{2}{c}{~~~~~\node{xs22}{X$'$}}\\[5ex]
                          &                     & \node{comp}{Komplement} & \node{x2}{X}\\
\end{tabular}
\nodeconnect{xp2}{spec}\nodeconnect{xp2}{xs2}%
\nodeconnect{xs2}{adj}\nodeconnect{xs2}{xs22}%
\nodeconnect{xs22}{comp}\nodeconnect{xs22}{x2}}
\caption{\label{Abb-GB-Min-Max}Minimaler und maximaler Ausbau von Phrasen}
\end{figure}
Beispiele für den minimalen Ausbau stellen die Determinatorenphrasen in
Abbildung~\ref{Abb-das-schoene-Bild-von-Paris}, der Baum für \emph{proud} in
Abbildung~\ref{Abbildung-AP} dar.
Adjunkte sind optional, weshalb es nicht unbedingt ein X$'$ mit Adjunkttochter geben muss. Für manche
Kategorien gibt es keinen Spezifikator, oder er ist optional.
%(Zusätzliche Regel nötig $\overline{\overline{\mbox{X}}} \rightarrow \xbar$)
Zusätzlich zu den eingezeichneten Verzweigungen werden mitunter Adjunkte an XP und Kopfadjunkte an X
zugelassen.

Die unären Verzweigungen bei Determinatoren sind unschön, aber konsequent.\footnote{
  Zu alternativen Varianten der \xbart, die ohne elaborierte Strukturen für Determinatoren
  auskommen, siehe auch \citew{Muysken82a}.
}
Die unären Verzweigungen für die NP für \emph{Paris} in
Abbildung~\ref{Abb-das-schoene-Bild-von-Paris} mögen ebenfalls Befremden auslösen, erscheinen aber
bei Berücksichtigung folgender stärker ausgebauter Nominalphrasen plausibler:
\eal
\ex das Paris der dreißiger Jahre
\ex die Maria aus Hamburg
\zl
Die unären Projektionen sind dennoch hässlich, was uns hier aber nicht weiter beunruhigen soll, denn
wir haben bei der Diskussion der Lexikoneinträge in (\ref{Lexikon-Projektion}) gesehen, dass sich unär verzweigende
Knoten zu einem großen Teil vermeiden lassen und dass es sogar erstrebenswert ist, diese Strukturen
zu vermeiden, da man sich ansonsten unechte Mehrdeutigkeiten einhandeln würde. In den kommenden
Kapiteln werden wir Ansätze wie Kategorialgrammatik und HPSG kennenlernen, die keine unären
Regeln für Adjektive und Nomina annehmen. 


\fi
