\documentclass[10pt,a3paper]{article}

\usepackage[margin=2cm]{geometry}

\usepackage{booktabs}
\usepackage{color}
\usepackage{soul}
\usepackage[linecolor=gray,backgroundcolor=yellow!50,textsize=tiny]{todonotes}
\usepackage[linguistics]{forest}
\usepackage{styles/my-gb4e-slides}

\usepackage{avm}

\usepackage[maxbibnames=99,
  maxcitenames=2,
  uniquelist=false,
  backend=biber,
  doi=false,
  url=false,
  isbn=false,
  bibstyle=biblatex-sp-unified,
  citestyle=sp-authoryear-comp]{biblatex}

\addbibresource{register.bib}

\newcommand{\eg}{e.\,g.}
\newcommand{\Eg}{E.\,g.}

\definecolor{rot}{rgb}{0.7,0.2,0.0}
\newcommand{\rot}[1]{\textcolor{rot}{#1}}
\definecolor{blau}{rgb}{0.1,0.2,0.7}
\newcommand{\blau}[1]{\textcolor{blau}{#1}}
\definecolor{gruen}{rgb}{0.0,0.7,0.2}
\newcommand{\gruen}[1]{\textcolor{gruen}{#1}}
\definecolor{grau}{rgb}{0.6,0.6,0.6}
\newcommand{\grau}[1]{\textcolor{grau}{#1}}
\definecolor{orongsch}{RGB}{255,165,0}
\newcommand{\orongsch}[1]{\textcolor{orongsch}{#1}}
\definecolor{tuerkis}{RGB}{63,136,143}
\definecolor{braun}{RGB}{108,71,65}
\newcommand{\tuerkis}[1]{\textcolor{tuerkis}{#1}}
\newcommand{\braun}[1]{\textcolor{braun}{#1}}

\newcommand*{\mybox}[1]{\framebox{#1}}

\forestset{
  decide/.style={draw, chamfered rectangle, inner sep=2pt},
  finall/.style={rounded corners, fill=gray, text=white},
  intrme/.style={draw, rounded corners},
  yes/.style={edge label={node[near end, above, sloped, font=\scriptsize]{Ja}}},
  no/.style={edge label={node[near end, above, sloped, font=\scriptsize]{Nein}}}
}

\newcommand{\Sub}[1]{\ensuremath{_{\text{#1}}}}
\newcommand{\Up}[1]{\ensuremath{^{\text{#1}}}}
\newcommand{\UpSub}[2]{\ensuremath{^{\text{#1}}_{\text{#2}}}}
\newcommand{\Zeile}{\vspace{\baselineskip}}
\newcommand{\Halbzeile}{\vspace{0.5\baselineskip}}
\newcommand{\Viertelzeile}{\vspace{0.25\baselineskip}}


\author{Roland Schäfer}
\title{HPSG: Musterlösung 5}
\date{\today}

\usepackage{fontspec}
\defaultfontfeatures{Ligatures=TeX,Numbers=OldStyle, Scale=MatchLowercase}
\setmainfont{Linux Libertine O}
\setsansfont{Linux Biolinum O}

\avmfont{\sc}
\avmsortfont{\it}
\avmvalfont{\it}

\pagestyle{empty}

\setlength\parindent{0pt}

\begin{document}

\maketitle

\thispagestyle{empty}

\section{Material}

Es geht hier ausschließlich um Satz (\ref{ex:satz}), wobei \textit{Kiki} und \textit{Die Strudlhofstiege} als unanalysierte Eigennamen (= NPs) behandelt werden sollen.

\begin{exe}
  \ex Kiki liest Die Strudlhofstiege.\label{ex:satz}
\end{exe}

Diese Musterlösung basiert auf Musterlösung 4.
Wir überlegen nur, was hinzugefügt werden muss bzw. was sich ändert.
Die Kommentare zur Lösung 4 werden gelöscht, und alle hinzugefügten Kommentare beziehen sich nur auf die Änderungen durch die Argumentextraktion.
In der gesamten Analyse wird hier die Semantik weggelassen.
Da aber immer der \textsc{local}-Wert über \textsc{dsl} und \textsc{slash} geteilt wird, wäre \textsc{cont} neben \textsc{cat} überall dabei und könnte die erwartete Struktur aufbauen.

\section{Lexikoneinträge}\label{sec:lex}

\begin{avm}
  \[ \asort{word}
    phon & \<\rm\it Kiki $\vee$ Die Strudlhofstiege\> \\
    loc|cat & \[
      head & \[ \asort{noun}
        cas & nom $\vee$ acc\\
        num & sg \\
      \]\\
      subcat & \<\> \\
    \] \\
  \]
\end{avm}

\begin{avm}
  \[ \asort{word}
    phon & \<\rm\it liest\> \\
    loc|cat & \[
      head & \[ \asort{verb}
        vform & fin \\
        initial & $-$ \\
      \]\\
      subcat & \<\[
        loc|cat & \[
          head & \[ \asort{noun}
            cas & nom \\
            num & sg \\
          \]\\
          subcat & \<\> \\
        \] \\
      \],
      \[
        loc|cat & \[
          head & \[ \asort{noun}
            cas & acc \\
          \]\\
          subcat & \<\> \\
        \] \\
      \]
      \> \\
    \] \\
  \]
\end{avm}

\section{V1-LR als DLR}\label{sec:v1lr}

\begin{avm}
  \[ \asort{lex-rule}
    loc|cat & \[
      head & \tuerkis{\[ \asort{verb}
        vform & fin \\
        \rot{initial} & \rot{$+$} \\
      \]}\\
      \gruen{subcat} & \<\[ loc|cat & \[
        head & \gruen{\[ \asort{verb}
          \blau{dsl} & \blau{\@1} \\
        \]}\\
        \gruen{subcat} & \gruen{\<\>} \\
      \]
      \]\> \\
    \] \\
    lex-dtr & \[
      \blau{loc} & \blau{\@1 \[ cat|head & \[ \asort{verb}
        vform & fin \\
        \orongsch{initial} & \orongsch{$-$} \\
      \]
    \]}
    \]
  \]
\end{avm}

\subsection{Spuren}

\subsubsection{Verbspur}

\begin{avm}
  \[ phon & \<\> \\
    loc  & \@{100} \[cat|head|dsl & \@{100}\] \\
  \]
\end{avm}

\subsubsection{Argumentspur}

\begin{avm}
	\[\asort{word}
		phon & \<\> \\
		loc  & \@{200} \\
		nonloc & \[slash & \<\@{200} \>\]
	\]
\end{avm}


\section{Aufbau des Satzes}

\subsection{Anwendung der V1-LR auf das lexikalische Verb}

\begin{avm}
  \[ \asort{lex-rule}
    phon & \<\rm\it liest\> \\
    loc|cat & \[
      head & \tuerkis{\[ \asort{verb}
        vform & fin \\
        \rot{initial} & \rot{$+$} \\
      \]}\\
      \gruen{subcat} & \<\[ loc|cat & \[
        head & \gruen{\[ \asort{verb}
          \blau{dsl} & \blau{\@1} \\
        \]}\\
        \gruen{subcat} & \gruen{\<\>} \\
      \]
      \]\> \\
    \] \\
    lex-dtr & \[ \asort{word}
    phon & \<\rm\it liest\> \\
    \blau{loc} & \blau{\@1 \[
      cat & \[
        head & \[ \asort{verb}
          vform & fin \\
          \orongsch{initial} & \orongsch{$-$} \\
        \]\\
        subcat & \<\[
          loc|cat & \[
            head & \[ \asort{noun}
              cas & nom \\
              num & sg \\
            \]\\
            subcat & \<\> \\
          \] \\
        \],
        \[
          loc|cat & \[
            head & \[ \asort{noun}
              cas & acc \\
            \]\\
            subcat & \<\> \\
          \] \\
        \]
        \> \\
      \] \\
    \]}
    \]
  \]
\end{avm}

\subsection{Verbindung des V1-Verbs mit der VP}\label{sec:top}

Den \textsc{phon}-Beitrag habe ich hier geändert zu \textit{Liest \_\_\_}, um anzuzeigen, dass über die Lautgestalt der gespurten VP eigentlich noch nichts bekannt ist.
(Das hat nichts mit der Argumentextraktion zu tun, und man hätte es auch schon auf der letzten Musterlösung so machen können.)

\Zeile

\begin{avm}
  \[ \asort{hd-arg-phrase}
    phon & \<\rm\it liest \_\_\_ \> \\
    cat & \[
      head & \tuerkis{\@3} \\
      subcat & \rot{\@4 \<\>} \\
    \] \\
    hd-dtr  & 
      \[ \asort{lex-rule}
        phon & \<\rm\it liest\> \\
        loc|cat & \[
          head & \tuerkis{\@3 \[ \asort{verb}
            vform & fin \\
            initial & $+$ \\
          \]}\\
          subcat & \rot{\@4}$\oplus$\<\orongsch{\@2}\> \\
        \] \\
        lex-dtr & \[ \asort{word}
        phon & \<\rm\it liest\> \\
        loc & \blau{\@1} 
        \]
    \] \\
    nhd-dtr &
      \orongsch{\@2\[
        loc|cat & \[
          head & \[ \asort{verb}
            \blau{dsl} & \blau{\@1
            \[
            cat & \[
              head & \[ \asort{verb}
                vform & fin \\
                initial & $-$ \\
              \]\\
              subcat & \< NP\Sub{nom}, NP\Sub{acc} \> \\
            \] \\
          \]}\\
          \]\\
          subcat & \<\> \\
        \]
      \]}
  \]
\end{avm}

\Zeile

Es gilt wie bei der letzten Musterlösung:
Wie auch immer wir die gespurte VP aufbauen, sie muss mit \orongsch{\mybox{2}} unifizieren.

\subsection{Aufbau der gespurten VP}\label{sec:vp}

\subsubsection{Verbindung der Verbspur mit \textit{Die Strudlhofstiege}}

\begin{avm}
  \[ \asort{hd-arg-phr}
    phon & \<\rm\it Die Strudlhofstiege \_ \> \\
    loc|cat & \[
      head & \rot{\@{105}} \\
      subcat & \braun{\@{102}}\\
    \]\\
    hd-dtr & \[
      phon & \<\> \\
      \tuerkis{loc}   & \tuerkis{\@{100} \[cat & \[
        head & \rot{\@{105} \[ dsl & \@{100} \]} \\
      subcat & \braun{\@{102}} $\oplus$ \< \@{101} \> \\
    \]
    \]} \\
    \]\\
    nhd-dtr & \tuerkis{\@{101}} \orongsch{\[\asort{word}
      phon & \<\rm\it Die Strudlhofstiege\> \\
      loc|cat & \[
        head & \[ \asort{noun}
          cas & acc\\
          num & sg \\
        \]\\
        subcat & \<\> \\
      \] \\
    \]}\\
  \]
\end{avm}

\subsubsection{Verbindung mit der Argumentspur}

Es ändert sich nicht viel gegenüber der Verbindung mit \textit{Kiki}.
Weil statt einer vollen NP ein Spur die Valenzstelle saturiert und die Spur extrem unterspezifiziert ist, kann jetzt nicht mehr eine \textsc{NP\Sub{nom}} auf der \textsc{subcat} vermutet werden.
Wir wissen eigentlich nur, dass dort \textbf{irgendein} \textsc{local}-Wert stehen muss, der mit \mybox{200} identifiziert wird.

\Zeile

Wichtig ist, dass die Spur selber einen \textsc{nonloc|slash} (identifiziert als \mybox{200}) mitbringt, der auf dem \textsc{nonloc|slash} der Mutterphrase aufgesammelt werden muss.
Er wird im Folgenden auch bei allen einschließenden Phrasen mitgenommen.

\Zeile

\begin{avm}
  \[ \asort{hd-arg-phr}
    phon & \<\rm\it \_ Die Strudlhofstiege \_ \> \\
    loc|cat & \[
      head & \rot{\@{105}} \\ 
      subcat & \braun{\@{103}\ \<\>}\\
    \]\\
    nonloc & \[ slash & \<\@{200}\> \] \\
    hd-dtr &
    \[ \asort{hd-arg-phr}
      phon & \<\rm\it Die Strudlhofstiege \_ \> \\
      loc|cat & \[
        head & \rot{\@{105}} \\
        subcat & \braun{\@{102} \@{103}$\oplus$\<\gruen{\@{104}}\>}\\
      \]\\
      hd-dtr & \[
        phon & \<\> \\
        \tuerkis{loc}   & \tuerkis{\@{100} \[cat & \[
        head & \rot{\@{105} \[ dsl & \@{100} \]} \\
        subcat & \braun{\@{102}} $\oplus$ \< \@{101} \> \\
      \]
      \]} \\
      \]\\
      nhd-dtr & \tuerkis{\@{101}} \orongsch{\[\asort{word}
        phon & \<\rm\it Die Strudlhofstiege\> \\
        loc|cat & \[
          head & \[ \asort{noun}
            cas & acc\\
            num & sg \\
          \]\\
          subcat & \<\> \\
        \] \\
      \]}\\
    \]\\
    nhd-dtr & \gruen{\@{104}} \[ \asort{word}
      phon & \<\ \> \\
      loc & \@{200} \\
      nonloc & \[ slash & \<\@{200}\> \] \\
    \]
  \]
\end{avm}

\newpage

\subsection{Unifizierung der VP mit dem bewegten Verb}

\subsubsection{Abkürzung der NPs in der doppelt gespurten VP}

Da wir nicht wissen, ob die Argumentspur ein Nominativ (oder eine NP) ist, können wir nur zur Struktur \mybox{104} vereinfachen.
Die \textsc{NP\Sub{acc}} können wir wie in der letzten Lösung behandeln.

\Zeile

\begin{avm}
  \[ \asort{hd-arg-phr}
    phon & \<\rm\it \_ Die Strudlhofstiege \_ \> \\
    loc|cat & \[
      head & \rot{\@{105}} \\ 
      subcat & \braun{\@{103}\ \<\>}\\
    \]\\
    nonloc & \[ slash & \@{200} \] \\
    hd-dtr &
    \[ \asort{hd-arg-phr}
      phon & \<\rm\it Die Strudlhofstiege \_ \> \\
      loc|cat & \[
        head & \rot{\@{105}} \\
        subcat & \braun{\@{102} \@{103}$\oplus$\<\gruen{\@{104}}\>}\\
      \]\\
      hd-dtr & \[
        \tuerkis{loc}   & \tuerkis{\@{100} \[cat & \[
        head & \rot{\@{105} \[ dsl & \@{100} \]} \\
        subcat & \braun{\@{102}} $\oplus$ \< \@{101} \> \\
      \]
      \]} \\
      \]\\
      nhd-dtr & \tuerkis{\@{101} \textup{\textsc{NP\Sub{acc}}}}\\
    \]\\
    nhd-dtr & \gruen{\@{104} \[ loc & \@{200} \\
	    nonloc & \[ slash & \@{200}\] \\
    \] }
  \]
\end{avm}

\subsubsection{Umstellung | \textsc{NP\Sub{acc}}}

Wie in der letzten Lösung verlagern wir die Darstellung der Information in \mybox{101} auf die \textsc{subcat} der Verbspur, wo wir sie am ehesten interpretieren können.

\Zeile

\begin{avm}
  \[ \asort{hd-arg-phr}
    phon & \<\rm\it \_ Die Strudlhofstiege \_ \> \\
    loc|cat & \[
      head & \rot{\@{105}} \\ 
      subcat & \braun{\@{103}\ \<\>}\\
    \]\\
    nonloc & \[ slash & \@{200}\] \\
    hd-dtr &
    \[ \asort{hd-arg-phr}
      phon & \<\rm\it Die Strudlhofstiege \_ \> \\
      loc|cat & \[
        head & \rot{\@{105}} \\
	subcat & \braun{\@{102} \@{103}$\oplus$\<\@{104} \[
		loc &\@{200}  \\
	\] \>}\\
      \]\\
      hd-dtr & \[
        \tuerkis{loc}   & \tuerkis{\@{100} \[cat & \[
        head & \rot{\@{105} \[ dsl & \@{100} \]} \\
        subcat & \braun{\@{102}} $\oplus$ \< \@{101} \textup{\textsc{NP\Sub{acc}}} \> \\
      \]
      \]} \\
      \]\\
    \]\\
  \]
\end{avm}

\subsubsection{Umstellung | Information der Spur}

Und das Gleiche machen wir mit der (stark reduzierten) Information unter \mybox{104}, die von der Spur kommt.

\Zeile

\begin{avm}
  \[ \asort{hd-arg-phr}
    phon & \<\rm\it \_ Die Strudlhofstiege \_ \> \\
    loc|cat & \[
      head & \rot{\@{105}} \\ 
      subcat & \braun{\<\>}\\
    \]\\
    nonloc & \[ slash & \@{200}\] \\
    hd-dtr &
    \[ \asort{hd-arg-phr}
      phon & \<\rm\it Die Strudlhofstiege \_ \> \\
      loc|cat & \[
        head & \rot{\@{105}} \\
      \]\\
      hd-dtr & \[
        \tuerkis{loc}   & \tuerkis{\@{100} \[cat & \[
        head & \rot{\@{105} \[ dsl & \@{100} \]} \\
	subcat &  \< \@{104} \[ loc & \@{200}\],
	\@{101} \textup{\textsc{NP\Sub{acc}}} \> \\
      \]
      \]} \\
      \]\\
    \]\\
  \]
\end{avm}

\subsubsection{Reduktion der gespurten VP aufs Wesentliche}

\begin{avm}
  \[ \asort{hd-arg-phr}
    phon & \<\rm\it \_ Die Strudlhofstiege \_ \> \\
    loc|cat & \[
      head & \rot{\@{105} \[ dsl & \@{100} \]} \\ 
      subcat & \braun{\<\>}\\
    \]\\
    nonloc & \[ slash & \<\@{200}\> \] \\
    hd-dtr &
    \[ hd-dtr & \[
        \tuerkis{loc}   & \tuerkis{\@{100} \[cat & \[
        head & \rot{\@{105}} \\
	subcat & \< \@{104} \[
		loc & \@{200}
	\], \@{101} \textup{\textsc{NP\Sub{acc}}} \> \\
      \]
      \]} \\
      \]\\
    \]\\
  \]
\end{avm}

\Zeile

Wieder gilt, dass diese Struktur mit dem \textsc{subcat}-Eintrag des bewegten Verbs aus Abschnitt~\ref{sec:top} unifiziert.
Im Gegensatz zur Situation mit einer overten Subjekt-NP bekommen wir in diesem Fall allerdings erst durch die Unifizierung mit dem bewegten Verb die Anforderung, dass das erste Element der \textsc{subcat} der Verbspur eine Nominativ-NP sein muss.

\Zeile

\begin{avm}
  \[
    loc|cat & \[
      head & \[ \asort{verb}
        dsl & \@1
        \[
        cat & \[
          head & \[ \asort{verb}
            vform & fin \\
            initial & $-$ \\
          \]\\
          subcat & \< NP\Sub{nom}, NP\Sub{acc} \> \\
        \] \\
      \]\\
      \]\\
      subcat & \<\> \\
    \]
  \]
\end{avm}

\Zeile

Beachten Sie, dass die folgenden beiden Strukturen unifizierbar sind, solange wir (wie hier) über \mybox{200} noch rein gar nichts gesagt haben.
Einerseits die Information über die Spur, die auf der \textsc{subcat} der Spur steht: 

\Zeile

\begin{avm}
  \@{104}\ \[loc & \@{200}\]
\end{avm}

\Zeile

Und die Anforderung des bewegten Verbs, dass es eine \textsc{NP\Sub{nom}} als erstes Element auf der \textsc{subcat} hat:

\Zeile

\begin{avm}
  \[loc & \[
    cat & \[ head & \[ \asort{noun}
    cas & nom \\
    num & sg \\
  \] \\
    subcat & \<\> \\
  \]
  \]
  \]
\end{avm}

\Zeile

Wir erhalten ganz einfach:

\Zeile

\begin{avm}
  \@{104}\ \[loc & \@{200} \[
    cat & \[ head & \[ \asort{noun}
    cas & nom \\
    num & sg \\
  \] \\
    subcat & \<\> \\
  \]
  \]
  \]
\end{avm}

\subsubsection{Kombination der V1-VP}

Hier ändert sich kaum etwas zur letzten Lösung, zumal wir inzwischen wissen, dass die Argumentspur die einer Nominativ-NP sein muss.
Wir könnten bei \mybox{104} jetzt an sich auch wieder \textsc{NP\Sub{nom}} angeben.
Und der \textsc{nonloc|slash} wird natürlich weiter mitgeschleppt.
Er zeigt jetzt auf den \textsc{local}-Wert \mybox{200}, der eine nicht weiter spezifizierte Nominativ-NP (im Singular) beschreibt.

\Zeile

\begin{avm}
  \[ \asort{hd-arg-phrase}
    phon & \<\rm\it Liest \_ Die Strudlhofstiege \_\> \\
    loc|cat & \[
      head & \tuerkis{\@3} \\
      subcat & \rot{\@4 \<\>} \\
    \] \\
    nonloc & \[ slash & \@{200}\] \\
    hd-dtr  & 
      \[ \asort{lex-rule}
        phon & \<\rm\it liest\> \\
        loc|cat & \[
          head & \tuerkis{\@3 \[ \asort{verb}
            vform & fin \\
            initial & $+$ \\
          \]}\\
          subcat & \rot{\@4}$\oplus$\<\orongsch{\@2}\> \\
        \] \\
        lex-dtr & \[ \asort{word}
        phon & \<\rm\it liest\> \\
        loc & \blau{\@1} 
        \]
    \] \\
    nhd-dtr &
      \orongsch{\@2\[
        phon & \<\rm\it \_ Die Strudlhofstiege \_\> \\
        loc|cat & \[
          head & \[ \asort{verb}
            \blau{dsl} & \blau{\@1
            \[
            cat & \[
              head & \[ \asort{verb}
                vform & fin \\
                initial & $-$ \\
              \]\\
	      subcat & \< \@{104} \[
		      loc & \@{200} \[
			      head & \[
				      case & nom \\
				      num & sg \\
			      \] \\
		      \] \\
	      \], \@{101} NP\Sub{acc} \> \\
            \] \\
          \]}\\
          \]\\
          subcat & \<\> \\
        \]
      \]}
  \]
\end{avm}

\subsubsection{Bau der Head-Filler Phrase}

Das wäre dann die Basis-Hausaufgabe.
Hier ist der Anfang:

\Zeile

\begin{avm}
  \[ \asort{hd-filler-phr}
    phon & \<\rm\it Kiki liest \_ Die Strudlhofstiege \_ \> \\
%
    loc|cat & \[
      head & \@{3} \\
      subcat & \@{4}\<\> \\ 
    \] \\
    nonloc|slash & \<\> \\
%
  filler-dtr & \[ \asort{word}
    phon & \<\rm\it Kiki \> \\
    loc \@{200} cat & \[
      head & \[ \asort{noun}
        cas & nom \\
        num & sg \\
      \]\\
      subcat & \<\> \\
    \] \\
    nonloc|slash & \<\> \\
  \] \\
%
    hd-dtr & \[ \asort{hd-arg-phrase}
    phon & \<\rm\it Liest \_ Die Strudlhofstiege \_\> \\
    loc|cat & \[
      head & \tuerkis{\@3} \\
      subcat & \rot{\@4 \<\>} \\
    \] \\
    nonloc & \[ slash & \@{200}\] \\
    hd-dtr  & 
      \[ \asort{lex-rule}
        phon & \<\rm\it liest\> \\
        loc|cat & \[
          head & \tuerkis{\@3 \[ \asort{verb}
            vform & fin \\
            initial & $+$ \\
          \]}\\
          subcat & \rot{\@4}$\oplus$\<\orongsch{\@2}\> \\
        \] \\
        lex-dtr & \[ \asort{word}
        phon & \<\rm\it liest\> \\
        loc & \blau{\@1} 
        \]
    \] \\
    nhd-dtr &
      \orongsch{\@2\[
        phon & \<\rm\it \_ Die Strudlhofstiege \_\> \\
        loc|cat & \[
          head & \[ \asort{verb}
            \blau{dsl} & \blau{\@1
            \[
            cat & \[
              head & \[ \asort{verb}
                vform & fin \\
                initial & $-$ \\
              \]\\
	      subcat & \< \@{104} \[
		      loc & \@{200} \[
			      head & \[
				      case & nom \\
				      num & sg \\
			      \] \\
		      \] \\
	      \], \@{101} NP\Sub{acc} \> \\
            \] \\
          \]}\\
          \]\\
          subcat & \<\> \\
        \]
      \]}
  \]
  \]
\end{avm}

\newpage

\section{Zusatzaufgaben}

\begin{enumerate}
  \item Überlegen Sie, wie man einen einfachen Relativsatz modellieren könnte.
    Es geht um, sowas wie das Fettgedruckte in (\ref{ex:rel0}).
    \begin{enumerate}
      \item Was wird in einem Relativsatz wohin bewegt?
        Machen Sie sich also zuerst Gedanken über die deskriptiven Unterschiede zwischen VL-Sätzen, unabhängigen V2-Sätzen und Relativsätzen.
      \item Für die relevante Bewegung im Relativsatz verwendet man \textsc{nonloc|rel} statt \textsc{nonloc|slash}.
        Überlegen Sie, was darauf abgelegt werden muss, und überlegen Sie, wie der Filler-Mechanismus dazu aussehen muss.
        Was ist gleich wie, was ist anders als bei \textsc{slash}?
    \end{enumerate}
    \Halbzeile
  \item Was ist das Problem an der Modellierung von komplexeren Relativsätzen wie in (\ref{ex:rel})?
    Überlegen Sie insbesondere, was die Probleme mit der Aufteilung der Merkmalstruktur in lokale Information inkl.\ \textsc{head} und nichtlokale Information sein könnten.
    \Halbzeile
  \item Überlegen Sie, wie man Adjunktextraktion modellieren könnte. \textbf{(Ausdrücklich kein Prüfungsstoff!)}
    \begin{enumerate}
      \item Versuchen Sie es mit einer lexikalischen Regel, die aus einem Adverbial (wie \textit{gerne}) eine Spur macht.
      \item Versuchen Sie es mit einer allgemeinen Adjunktspur.
    \end{enumerate}
\end{enumerate}

\Zeile

\begin{exe}
  \ex\label{ex:rel0} das Buch, \textbf{das ich lese}
  \ex\label{ex:rel} das Buch, \textbf{dessen Inhalt ich kenne}
\end{exe}

\end{document}
