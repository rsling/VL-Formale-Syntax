\documentclass[10pt,a3paper]{article}

\usepackage[margin=2cm]{geometry}

\usepackage{booktabs}
\usepackage{color}
\usepackage{soul}
\usepackage[linecolor=gray,backgroundcolor=yellow!50,textsize=tiny]{todonotes}
\usepackage[linguistics]{forest}
\usepackage{styles/my-gb4e-slides}

\usepackage{avm}

\usepackage[maxbibnames=99,
  maxcitenames=2,
  uniquelist=false,
  backend=biber,
  doi=false,
  url=false,
  isbn=false,
  bibstyle=biblatex-sp-unified,
  citestyle=sp-authoryear-comp]{biblatex}

\addbibresource{register.bib}

\newcommand{\eg}{e.\,g.}
\newcommand{\Eg}{E.\,g.}

\definecolor{rot}{rgb}{0.7,0.2,0.0}
\newcommand{\rot}[1]{\textcolor{rot}{#1}}
\definecolor{blau}{rgb}{0.1,0.2,0.7}
\newcommand{\blau}[1]{\textcolor{blau}{#1}}
\definecolor{gruen}{rgb}{0.0,0.7,0.2}
\newcommand{\gruen}[1]{\textcolor{gruen}{#1}}
\definecolor{grau}{rgb}{0.6,0.6,0.6}
\newcommand{\grau}[1]{\textcolor{grau}{#1}}
\definecolor{orongsch}{RGB}{255,165,0}
\newcommand{\orongsch}[1]{\textcolor{orongsch}{#1}}
\definecolor{tuerkis}{RGB}{63,136,143}
\definecolor{braun}{RGB}{108,71,65}
\newcommand{\tuerkis}[1]{\textcolor{tuerkis}{#1}}
\newcommand{\braun}[1]{\textcolor{braun}{#1}}

\newcommand*{\mybox}[1]{\framebox{#1}}

\forestset{
  decide/.style={draw, chamfered rectangle, inner sep=2pt},
  finall/.style={rounded corners, fill=gray, text=white},
  intrme/.style={draw, rounded corners},
  yes/.style={edge label={node[near end, above, sloped, font=\scriptsize]{Ja}}},
  no/.style={edge label={node[near end, above, sloped, font=\scriptsize]{Nein}}}
}

\newcommand{\Sub}[1]{\ensuremath{_{\text{#1}}}}
\newcommand{\Up}[1]{\ensuremath{^{\text{#1}}}}
\newcommand{\UpSub}[2]{\ensuremath{^{\text{#1}}_{\text{#2}}}}
\newcommand{\Zeile}{\vspace{\baselineskip}}
\newcommand{\Halbzeile}{\vspace{0.5\baselineskip}}
\newcommand{\Viertelzeile}{\vspace{0.25\baselineskip}}


\author{Roland Schäfer}
\title{HPSG: Musterlösung 4}
\date{\today}

\usepackage{fontspec}
\defaultfontfeatures{Ligatures=TeX,Numbers=OldStyle, Scale=MatchLowercase}
\setmainfont{Linux Libertine O}
\setsansfont{Linux Biolinum O}

\avmfont{\sc}
\avmsortfont{\it}
\avmvalfont{\it}

\pagestyle{empty}

\setlength\parindent{0pt}

\begin{document}

\maketitle

\thispagestyle{empty}

\section{Material}

Es geht hier ausschließlich um Satz (\ref{ex:satz}), wobei \textit{Kiki} und \textit{Die Strudlhofstiege} als unanalysierte Eigennamen (= NPs) behandelt werden sollen.

\begin{exe}
  \ex Liest Kiki Die Strudlhofstiege?\label{ex:satz}
\end{exe}

In der gesamten Analyse wird hier die Semantik weggelassen.
Da aber immer der \textsc{local}-Wert über \textsc{dsl} geteilt wird, wäre \textsc{cont} neben \textsc{cat} überall dabei und könnte die erwartete Wirkung haben.

\section{Lexikoneinträge}\label{sec:lex}

\begin{avm}
  \[ \asort{word}
    phon & \<\rm\it Kiki $\vee$ Die Strudlhofstiege\> \\
    loc|cat & \[
      head & \[ \asort{noun}
        cas & nom $\vee$ acc\\
        num & sg \\
      \]\\
      subcat & \<\> \\
    \] \\
  \]
\end{avm}

\begin{avm}
  \[ \asort{word}
    phon & \<\rm\it liest\> \\
    loc|cat & \[
      head & \[ \asort{verb}
        vform & fin \\
        initial & $-$ \\
      \]\\
      subcat & \<\[
        loc|cat & \[
          head & \[ \asort{noun}
            cas & nom \\
            num & sg \\
          \]\\
          subcat & \<\> \\
        \] \\
      \],
      \[
        loc|cat & \[
          head & \[ \asort{noun}
            cas & acc \\
          \]\\
          subcat & \<\> \\
        \] \\
      \]
      \> \\
    \] \\
  \]
\end{avm}

\section{V1-LR als DLR}\label{sec:v1lr}

\begin{avm}
  \[ \asort{lex-rule}
    loc|cat & \[
      head & \tuerkis{\[ \asort{verb}
        vform & fin \\
        \rot{initial} & \rot{$+$} \\
      \]}\\
      \gruen{subcat} & \<\[ loc|cat & \[
        head & \gruen{\[ \asort{verb}
          \blau{dsl} & \blau{\@1} \\
        \]}\\
        \gruen{subcat} & \gruen{\<\>} \\
      \]
      \]\> \\
    \] \\
    lex-dtr & \[
      \blau{loc} & \blau{\@1 \[ cat|head & \[ \asort{verb}
        vform & fin \\
        \orongsch{initial} & \orongsch{$-$} \\
      \]
    \]}
    \]
  \]
\end{avm}

\Zeile

Der Input (\textsc{lex-dtr}) der Regel muss ein \blau{finites Verb} sein, das \orongsch{Verbletztstellung} verlangt.
Das Ergebnis der Anwendung dieser Regel ist ein \tuerkis{finites Verb}, das \rot{Verberststellung} verlangt.
Dieses V1-Verb hat eine für ein Verb atypische \gruen{Valenz}, weil es eine \gruen{VP} verlangt, aus der ein \blau{ein finites Verb} extrahiert wurde (\textsc{\blau{dsl}}), welches \orongsch{Verbletztstellung} verlangt.
Diese Anforderungen werden sichergestellt, weil der \blau{\textsc{local}}-Wert der lexikalischen Tochter strukturgeteilt mit dem der \blau{Verbspur (\textsc{dsl})} ist (\blau{\mybox{1}}).

\newpage

\section{Aufbau des Satzes}

\subsection{Anwendung der V1-LR auf das lexikalische Verb}

Die Farbgebung folgt der in Abschnitt~\ref{sec:v1lr}.

\Zeile

\begin{avm}
  \[ \asort{lex-rule}
    phon & \<\rm\it liest\> \\
    loc|cat & \[
      head & \tuerkis{\[ \asort{verb}
        vform & fin \\
        \rot{initial} & \rot{$+$} \\
      \]}\\
      \gruen{subcat} & \<\[ loc|cat & \[
        head & \gruen{\[ \asort{verb}
          \blau{dsl} & \blau{\@1} \\
        \]}\\
        \gruen{subcat} & \gruen{\<\>} \\
      \]
      \]\> \\
    \] \\
    lex-dtr & \[ \asort{word}
    phon & \<\rm\it liest\> \\
    \blau{loc} & \blau{\@1 \[
      cat & \[
        head & \[ \asort{verb}
          vform & fin \\
          \orongsch{initial} & \orongsch{$-$} \\
        \]\\
        subcat & \<\[
          loc|cat & \[
            head & \[ \asort{noun}
              cas & nom \\
              num & sg \\
            \]\\
            subcat & \<\> \\
          \] \\
        \],
        \[
          loc|cat & \[
            head & \[ \asort{noun}
              cas & acc \\
            \]\\
            subcat & \<\> \\
          \] \\
        \]
        \> \\
      \] \\
    \]}
    \]
  \]
\end{avm}

\subsection{Verbindung des V1-Verbs mit der VP}\label{sec:top}

Wir wissen jetzt, welche Bedingungen die gespurte VP erfüllen muss, denn das bewegte Verb verbindet sich über die normale Valenzabbindung mit ihr.
(Mit \textit{gespurte VP} ist stets die VP gemeint, die die Verbspur enthält.)
Wir beginnen mit einer leicht abgekürzten Version des bewegten Verbs (nur die \textsc{subcat} wurde abgekürzt), auf der \textbf{in der Darstellung} die \blau{\mybox{1}} auf den \blau{\textsc{loc|cat|head|dsl}-Wert der per Valenz geforderten gespurten VP} verschoben wurde.
Dank Strukturteilung ist das wirklich \textbf{nur} eine Frage der Darstellung.
Wir beginnen außerdem teilweise eine neue Farbgebung.

\Zeile

\begin{avm}
  \[ \asort{lex-rule}
    phon & \<\rm\it liest\> \\
    loc|cat & \[
      head & \[ \asort{verb}
        vform & fin \\
        initial & $+$ \\
      \]\\
      subcat & \<\orongsch{\[ loc|cat & \[
        head & \[ \asort{verb}
          \blau{dsl} & \blau{\@1
          \[
          cat & \[
            head & \[ \asort{verb}
              vform & fin \\
              initial & $-$ \\
            \]\\
            subcat & \< NP\Sub{nom}, NP\Sub{acc} \> \\
          \] \\
        \]}\\
        \]\\
        subcat & \<\> \\
      \]
    \]}\> \\
    \] \\
    lex-dtr & \[ \asort{word}
    phon & \<\rm\it liest\> \\
    loc & \blau{\@1} 
    \]
  \]
\end{avm}

\newpage

Das erlaubt es uns jetzt, zu sagen, wie die VP auf Satzebene aussehen muss.
Die \textsc{nhd-dtr} muss beschaffen sein wie auf der Valenz des bewegten Verbs beschrieben.
Wir können also einfach den gesamten \orongsch{\textsc{subcat}-Eintrag} als Spezifikation der Nichtkopf-Tochter ansetzen (inklusive des enthaltenen \blau{\textsc{dsl}}).
Die Kopftochter entspricht exakt dem oben spezifizierten bewegten Verb.
Also verschieben wir nochmals, aber wieder bleibt dank Strukturteilung alles überall erhalten, wir ändern nur die Darstellung.
Die \textsc{phon}-Werte stehen nur zur Illustration bzw.\ Orientierung.
Überall dort, wo die Phrase die Spur enthält, ist ein \_ in der Phon an ihrer Stelle eingefügt.
Auch das dient natürlich nur der Illustration.

\Zeile

\begin{avm}
  \[ \asort{hd-arg-phrase}
    phon & \<\rm\it Liest Kiki Die Strudlhofstiege \_\> \\
    cat & \[
      head & \tuerkis{\@3} \\
      subcat & \rot{\@4 \<\>} \\
    \] \\
    hd-dtr  & 
      \[ \asort{lex-rule}
        phon & \<\rm\it liest\> \\
        loc|cat & \[
          head & \tuerkis{\@3 \[ \asort{verb}
            vform & fin \\
            initial & $+$ \\
          \]}\\
          subcat & \rot{\@4}$\oplus$\<\orongsch{\@2}\> \\
        \] \\
        lex-dtr & \[ \asort{word}
        phon & \<\rm\it liest\> \\
        loc & \blau{\@1} 
        \]
    \] \\
    nhd-dtr &
      \orongsch{\@2\[
        phon & \<\rm\it Kiki Die Strudlhofstiege \_\> \\
        loc|cat & \[
          head & \[ \asort{verb}
            \blau{dsl} & \blau{\@1
            \[
            cat & \[
              head & \[ \asort{verb}
                vform & fin \\
                initial & $-$ \\
              \]\\
              subcat & \< NP\Sub{nom}, NP\Sub{acc} \> \\
            \] \\
          \]}\\
          \]\\
          subcat & \<\> \\
        \]
      \]}
  \]
\end{avm}

\Zeile

Wie auch immer wir die gespurte VP aufbauen, sie muss mit \orongsch{\mybox{2}} unifizieren.

\subsection{Aufbau der gespurten VP}\label{sec:vp}

Wir bauen jetzt die VP auf und gehen nur von der Spur aus.
Dabei unifizieren wir erstmal alles Mögliche (bei der Valenzabbindung) und schauen, ob das Richtige rauskommt.
Wir konstruieren die VP sozusagen hypothetisch und ermitteln so, ob sie dann in die Struktur aus Abschnitt~\ref{sec:top} passt. 
Wenn es dann passt, haben wir eine erfolgreiche Ableitung für den Satz.
Dieses Darstellung ist eventuell nachvollziehbarer als die statische (korrektere) Denkweise, bei der alles in einem Schritt unifiziert.

Die Spur sieht bekanntlich wiefolgt aus.

\Zeile

\begin{avm}
  \[ phon & \<\> \\
    \tuerkis{loc}   & \tuerkis{\@{100} \[cat|head|dsl & \@{100}\]} \\
  \]
\end{avm}

\Zeile

Um die Spur als Kopftochter mit \textit{Die Strudlhofstiege} als Nichtkopftochter verbinden zu können, müsste ein passender Eintrag auf der \textsc{subcat} der Spur sein.
Als Nichtkopftochter wird hier der \orongsch{Eintrag für \textit{Die Strudlhofstiege}} aus Abschnitt~\ref{sec:lex} im Akkusativ eingesetzt.
Das Resultat sähe also so aus, wobei hier noch nicht klar ist, ob die restliche \textsc{subcat} in \braun{\mybox{102}} leer ist oder nicht.

\Zeile

\begin{avm}
  \[ \asort{hd-arg-phr}
    phon & \<\rm\it Die Strudlhofstiege \_ \> \\
    loc|cat & \[
      head & \rot{\@{105}} \\
      subcat & \braun{\@{102}}\\
    \]\\
    hd-dtr & \[
      phon & \<\> \\
      \tuerkis{loc}   & \tuerkis{\@{100} \[cat & \[
        head & \rot{\@{105} \[ dsl & \@{100} \]} \\
      subcat & \braun{\@{102}} $\oplus$ \< \@{101} \> \\
    \]
    \]} \\
    \]\\
    nhd-dtr & \tuerkis{\@{101}} \orongsch{\[\asort{word}
      phon & \<\rm\it Die Strudlhofstiege\> \\
      loc|cat & \[
        head & \[ \asort{noun}
          cas & acc\\
          num & sg \\
        \]\\
        subcat & \<\> \\
      \] \\
    \]}\\
  \]
\end{avm}

\newpage

Diese Projektion können wir jetzt mit dem nächsten Argument \textit{Kiki} verbinden.
Dazu müsste \braun{\mybox{102}} noch mindestens einen Eintrag enthalten.
Achtung! Das Label \mybox{102} identifiziert die Konkatenation aus der beliebigen Liste \mybox{103} und der einelementigen Liste \mybox{104}.
Wir teilen \mybox{102} also noch weiter auf.
Weil wir ja schon wissen, was rauskommt, gehen wir davon aus, dass \mybox{103} leer ist und zeigen das auf der \textsc{subcat} der gesamten VP an.

\Zeile

\begin{avm}
  \[ \asort{hd-arg-phr}
    phon & \<\rm\it Kiki Die Strudlhofstiege \_ \> \\
    loc|cat & \[
      head & \rot{\@{105}} \\ 
      subcat & \braun{\@{103}\ \<\>}\\
    \]\\
    hd-dtr &
    \[ \asort{hd-arg-phr}
      phon & \<\rm\it Die Strudlhofstiege \_ \> \\
      loc|cat & \[
        head & \rot{\@{105}} \\
        subcat & \braun{\@{102} \@{103}$\oplus$\<\gruen{\@{104}}\>}\\
      \]\\
      hd-dtr & \[
        phon & \<\> \\
        \tuerkis{loc}   & \tuerkis{\@{100} \[cat & \[
        head & \rot{\@{105} \[ dsl & \@{100} \]} \\
        subcat & \braun{\@{102}} $\oplus$ \< \@{101} \> \\
      \]
      \]} \\
      \]\\
      nhd-dtr & \tuerkis{\@{101}} \orongsch{\[\asort{word}
        phon & \<\rm\it Die Strudlhofstiege\> \\
        loc|cat & \[
          head & \[ \asort{noun}
            cas & acc\\
            num & sg \\
          \]\\
          subcat & \<\> \\
        \] \\
      \]}\\
    \]\\
    nhd-dtr & \gruen{\@{104}} \[ \asort{word}
      phon & \<\rm\it Kiki \> \\
      loc|cat & \[
        head & \[ \asort{noun}
          cas & nom \\
          num & sg \\
        \]\\
        subcat & \<\> \\
      \] \\
    \]
  \]
\end{avm}

\subsection{Unifizierung der VP mit dem bewegten Verb}

Die Frage ist jetzt, ob das Ergebnis mit der geforderten Struktur aus Abschnitt~\ref{sec:top} zusammenpasst.
Dazu stellen wir das Endergebnis aus Abschnitt~\ref{sec:vp} ein bisschen um.
Insbesondere kürzen wir die NPs als \textsc{np\Sub{nom}} (\textit{Kiki}) und \textsc{NP\Sub{acc}} (\textit{Die Strudlhofstiege}) ab und kürzen dann die Struktur um die \textsc{hd-dtr} und \textsc{nhd-dtr}, die weiter oben in der Struktur keine Rolle mehr für unseren Zweck spielen, außer dass sie mit den Kopfmerkmalen Strukturen teilen.
\mybox{104} wird also als NP\Sub{nom} und \mybox{101} als NP\Sub{acc} abgekürzt.
Die Abkürzung ist natürlich nur eine Sache der Darstellung.
Wir erhalten:

\Zeile

\begin{avm}
  \[ \asort{hd-arg-phr}
    phon & \<\rm\it Kiki Die Strudlhofstiege \_ \> \\
    loc|cat & \[
      head & \rot{\@{105}} \\ 
      subcat & \braun{\@{103}\ \<\>}\\
    \]\\
    hd-dtr &
    \[ \asort{hd-arg-phr}
      phon & \<\rm\it Die Strudlhofstiege \_ \> \\
      loc|cat & \[
        head & \rot{\@{105}} \\
        subcat & \braun{\@{102} \@{103}$\oplus$\<\gruen{\@{104}}\>}\\
      \]\\
      hd-dtr & \[
        \tuerkis{loc}   & \tuerkis{\@{100} \[cat & \[
        head & \rot{\@{105} \[ dsl & \@{100} \]} \\
        subcat & \braun{\@{102}} $\oplus$ \< \@{101} \> \\
      \]
      \]} \\
      \]\\
      nhd-dtr & \tuerkis{\@{101} \textup{\textsc{NP\Sub{acc}}}}\\
    \]\\
    nhd-dtr & \gruen{\@{104} \textup{\textsc{NP\Sub{nom}}}}
  \]
\end{avm}

\Zeile

Wir können jetzt in der \textsc{hd-dtr} die \textsc{subcat} füllen und die \textsc{nhd-dtr} in der Darstellung entfernen, weil sie uns gerade nicht mehr im Detail interessiert.
Wir beginnen mit \mybox{101}:

\Zeile

\begin{avm}
  \[ \asort{hd-arg-phr}
    phon & \<\rm\it Kiki Die Strudlhofstiege \_ \> \\
    loc|cat & \[
      head & \rot{\@{105}} \\ 
      subcat & \braun{\@{103}\ \<\>}\\
    \]\\
    hd-dtr &
    \[ \asort{hd-arg-phr}
      phon & \<\rm\it Die Strudlhofstiege \_ \> \\
      loc|cat & \[
        head & \rot{\@{105}} \\
        subcat & \braun{\@{102} \@{103}$\oplus$\<\gruen{\@{104}}\>}\\
      \]\\
      hd-dtr & \[
        \tuerkis{loc}   & \tuerkis{\@{100} \[cat & \[
        head & \rot{\@{105} \[ dsl & \@{100} \]} \\
        subcat & \braun{\@{102}} $\oplus$ \< \@{101} \textup{\textsc{NP\Sub{acc}}} \> \\
      \]
      \]} \\
      \]\\
    \]\\
    nhd-dtr & \gruen{\@{104} \textup{\textsc{NP\Sub{nom}}}}
  \]
\end{avm}

\newpage

Und wir fahren fort mit \mybox{104}:

\Zeile

\begin{avm}
  \[ \asort{hd-arg-phr}
    phon & \<\rm\it Kiki Die Strudlhofstiege \_ \> \\
    loc|cat & \[
      head & \rot{\@{105}} \\ 
      subcat & \braun{\@{103}\ \<\>}\\
    \]\\
    hd-dtr &
    \[ \asort{hd-arg-phr}
      phon & \<\rm\it Die Strudlhofstiege \_ \> \\
      loc|cat & \[
        head & \rot{\@{105}} \\
        subcat & \braun{\@{102} \@{103}$\oplus$\<\gruen{\@{104} \textup{\textsc{NP\Sub{nom}}}}\>}\\
      \]\\
      hd-dtr & \[
        \tuerkis{loc}   & \tuerkis{\@{100} \[cat & \[
        head & \rot{\@{105} \[ dsl & \@{100} \]} \\
        subcat & \braun{\@{102}} $\oplus$ \< \@{101} \textup{\textsc{NP\Sub{acc}}} \> \\
      \]
      \]} \\
      \]\\
    \]\\
  \]
\end{avm}

\Zeile

Wir haben bereits entschieden, dass \mybox{103} leer sein muss.
Daher wissen wir, dass \mybox{102} nur \mybox{104} enthält.
Wir vereinfachen daher an drei Stellen und lösen \mybox{103} auf.

\Zeile

\begin{avm}
  \[ \asort{hd-arg-phr}
    phon & \<\rm\it Kiki Die Strudlhofstiege \_ \> \\
    loc|cat & \[
      head & \rot{\@{105}} \\ 
      subcat & \braun{\<\>}\\
    \]\\
    hd-dtr &
    \[ \asort{hd-arg-phr}
      phon & \<\rm\it Die Strudlhofstiege \_ \> \\
      loc|cat & \[
        head & \rot{\@{105}} \\
        subcat & \braun{\<\gruen{\@{104}}\>}\\
      \]\\
      hd-dtr & \[
        \tuerkis{loc}   & \tuerkis{\@{100} \[cat & \[
        head & \rot{\@{105} \[ dsl & \@{100} \]} \\
        subcat & \< \gruen{\@{104} \textup{\textsc{NP\Sub{nom}}}}, \@{101} \textup{\textsc{NP\Sub{acc}}} \> \\
      \]
      \]} \\
      \]\\
    \]\\
  \]
\end{avm}

\Zeile

Die \mybox{105} für \textsc{loc|cat|head} der gespurten VP taucht drei mal auf.
Wir schieben ihre Beschreibung von \textsc{hd-dtr|hd-dtr|loc|cat|head} ganz nach oben und entfernen die anderen Instanzen aus der Darstellung (nicht aus der eigentlichen Merkmalsbeschreibung).
\textsc{hd-dtr|loc|cat} kann damit ganz aus der Darstellung entfernt werden, weil alle seine relevanten Strukturteilungen auf \textsc{loc|cat} aufgegriffen werden.
\textsc{hd-dtr|phon} kann auch weg.

\Zeile

\begin{avm}
  \[ \asort{hd-arg-phr}
    phon & \<\rm\it Kiki Die Strudlhofstiege \_ \> \\
    loc|cat & \[
      head & \rot{\@{105} \[ dsl & \@{100} \]} \\ 
      subcat & \braun{\<\>}\\
    \]\\
    hd-dtr &
    \[ hd-dtr & \[
        \tuerkis{loc}   & \tuerkis{\@{100} \[cat & \[
        head & \rot{\@{105}} \\
        subcat & \< \gruen{\@{104} \textup{\textsc{NP\Sub{nom}}}}, \@{101} \textup{\textsc{NP\Sub{acc}}} \> \\
      \]
      \]} \\
      \]\\
    \]\\
  \]
\end{avm}

\Zeile

Jetzt können wir die \mybox{100} unter \textsc{loc|cat|head|dsl} darstellen und die \textsc{hd-dtr} aus der Darstellung nehmen.
Die Farben können auch weg.

\Zeile

\begin{avm}
  \[ \asort{hd-arg-phr}
    phon & \<\rm\it Kiki Die Strudlhofstiege \_ \> \\
    loc|cat & \[
      head & \@{105} \[ dsl & \@{100} \[
        cat & \[
          head & \@{105} \\
          subcat & \< \@{104} \textup{\textsc{NP\Sub{nom}}}, \@{101} \textup{\textsc{NP\Sub{acc}}} \> \\
        \]
      \]\] \\ 
      subcat & \<\>\\
    \]\\
  \]
\end{avm}

\Zeile

Die geforderte gespurte VP in Abschnitt~\ref{sec:top} sah folgendermaßen aus (s.\ dort die \textsc{subcat}):

\Zeile

\begin{avm}
  \[
    phon & \<\rm\it Kiki Die Strudlhofstiege \_\> \\
    loc|cat & \[
      head & \[ \asort{verb}
        dsl & \@1
        \[
        cat & \[
          head & \[ \asort{verb}
            vform & fin \\
            initial & $-$ \\
          \]\\
          subcat & \< NP\Sub{nom}, NP\Sub{acc} \> \\
        \] \\
      \]\\
      \]\\
      subcat & \<\> \\
    \]
  \]
\end{avm}

\Zeile

Die beiden Strukturen unifizieren ganz offensichtlich mit \mybox{1}=\mybox{100}.
Wir hatten bei der Spur nicht spezifiziert, dass sie ein Verbletzt-Verb ist (\textsc{head} [\textit{verb} \textsc{initial} $-$]), aber genau das wäre sie.
Und die gesamte gespurte VP hat dieselben Kopfmerkmale, also passt auch die Teilung \mybox{105}.
Der Punkt dabei ist, dass wir ganz oben auf der Satzebene legitimieren, dass wir die gespurte VP so aufgebaut haben, wie wir es getan haben.
Die hypothetisch angenommene Valenz der Spur erweist sich an diesem Punkt als korrekt, weil das bewegte Verb auf einem lexikalischen Verb basiert, das die passende Valenz hat.
Wow!

Das Ganze geschieht in der (theoretischen) HPSG nicht in solchen Schritten mit Hypothesen über die Verbspur (in einem implementierten Parser eventuell aber schon bzw.\ entfernt so ähnlich).
Diese Ableitung diente nur der Verdeutlichung, wie die Spur vom bewegten Verb ihren \textsc{local}-Wert durch die ganze Struktur hindurch (über \textsc{dsl}) aufgezwungen bekommt.
Alles, was die HPSG sagt, ist, dass die Struktur \textit{Liest\Sub{1} Kiki Die Strudlhofstiege t\Sub{1}} ein Satz ist, weil auf Basis der Lexikoneinträge und Regeln über die entsprechenden Strukturteilungen die Wörter entsprechend unifizieren.

\subsection{Aufgabe}

Zeigen Sie parallel zu dieser Musterlösung, dass (\ref{ex:meeep}) nicht ableitbar ist.
Behandeln Sie dabei \textit{des Buchs} wie \textit{Die Strudlhofstiege} als eine nicht weiter analysierte NP, allerdings im Genitiv.
Kürzen Sie die nominalen Einträge auf der \textsc{subcat} des Verbs von Anfang an stark ab.

\begin{exe}
  \ex[*]{Liest\Sub{1} Kiki des Buchs t\Sub{1}.\label{ex:meeep}}
\end{exe}

\end{document}

