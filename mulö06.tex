\documentclass[10pt,a3paper]{article}

\usepackage[margin=2cm]{geometry}

\usepackage{booktabs}
\usepackage{color}
\usepackage{soul}
\usepackage[linecolor=gray,backgroundcolor=yellow!50,textsize=tiny]{todonotes}
\usepackage[linguistics]{forest}
\usepackage{styles/my-gb4e-slides}

\usepackage{avm}

\usepackage[maxbibnames=99,
  maxcitenames=2,
  uniquelist=false,
  backend=biber,
  doi=false,
  url=false,
  isbn=false,
  bibstyle=biblatex-sp-unified,
  citestyle=sp-authoryear-comp]{biblatex}

\addbibresource{register.bib}

\newcommand{\eg}{e.\,g.}
\newcommand{\Eg}{E.\,g.}

\definecolor{rot}{rgb}{0.7,0.2,0.0}
\definecolor{blau}{rgb}{0.1,0.2,0.7}
\definecolor{gruen}{rgb}{0.0,0.7,0.2}
\definecolor{grau}{rgb}{0.6,0.6,0.6}
\definecolor{orongsch}{RGB}{255,165,0}
\definecolor{tuerkis}{RGB}{63,136,143}
\definecolor{braun}{RGB}{108,71,65}
\definecolor{pinku}{RGB}{255,180,200}
\definecolor{violett}{RGB}{150,50,250}

% \definecolor{rot}{rgb}{0,0,0}
% \definecolor{blau}{rgb}{0,0,0}
% \definecolor{gruen}{rgb}{0,0,0}
% \definecolor{grau}{rgb}{0,0,0}
% \definecolor{orongsch}{RGB}{0,0,0}
% \definecolor{tuerkis}{RGB}{0,0,0}
% \definecolor{braun}{RGB}{0,0,0}
% \definecolor{pinku}{RGB}{0,0,0}
% \definecolor{violett}{RGB}{0,0,0}

\newcommand{\rot}[1]{\textcolor{rot}{#1}}
\newcommand{\blau}[1]{\textcolor{blau}{#1}}
\newcommand{\gruen}[1]{\textcolor{gruen}{#1}}
\newcommand{\grau}[1]{\textcolor{grau}{#1}}
\newcommand{\orongsch}[1]{\textcolor{orongsch}{#1}}
\newcommand{\tuerkis}[1]{\textcolor{tuerkis}{#1}}
\newcommand{\braun}[1]{\textcolor{braun}{#1}}
\newcommand{\pinku}[1]{\textcolor{pinku}{#1}}
\newcommand{\violett}[1]{\textcolor{violett}{#1}}

\newcommand*{\mybox}[1]{\framebox{#1}}

\forestset{
  decide/.style={draw, chamfered rectangle, inner sep=2pt},
  finall/.style={rounded corners, fill=gray, text=white},
  intrme/.style={draw, rounded corners},
  yes/.style={edge label={node[near end, above, sloped, font=\scriptsize]{Ja}}},
  no/.style={edge label={node[near end, above, sloped, font=\scriptsize]{Nein}}}
}

\newcommand{\Sub}[1]{\ensuremath{_{\text{#1}}}}
\newcommand{\Up}[1]{\ensuremath{^{\text{#1}}}}
\newcommand{\UpSub}[2]{\ensuremath{^{\text{#1}}_{\text{#2}}}}
\newcommand{\Zeile}{\vspace{\baselineskip}}
\newcommand{\Halbzeile}{\vspace{0.5\baselineskip}}
\newcommand{\Viertelzeile}{\vspace{0.25\baselineskip}}


\author{Roland Schäfer}
\title{HPSG: Lösungsansatz (Musterlösung 6)}
\date{\today}

\usepackage{fontspec}
\defaultfontfeatures{Ligatures=TeX,Numbers=OldStyle, Scale=MatchLowercase}
\setmainfont{Linux Libertine O}
\setsansfont{Linux Biolinum O}

\avmfont{\sc}
\avmsortfont{\it}
\avmvalfont{\it}

\pagestyle{empty}

\setlength\parindent{0pt}

\begin{document}

\maketitle

\thispagestyle{empty}

\section{Material}

Es geht hier ausschließlich um Satz (\ref{ex:satz}).

\begin{exe}
  \ex Meistens filmt Dan.\label{ex:satz}
\end{exe}

\section{Lexikoneinträge}\label{sec:lex}

\begin{avm}
  \[ \asort{word}
    phon & \<\rm\it Dan \> \\
    loc|cat & \[
      head & \[ \asort{noun}
        cas & nom $\vee$ acc $\vee$ dat\\
        num & sg \\
      \]\\
      subcat & \<\> \\
    \] \\
  \]
\end{avm}

\begin{avm}
  \[ \asort{word}
    phon & \<\rm\it filmt\> \\
    loc|cat & \[
      head & \[ \asort{verb}
        vform & fin \\
        initial & $-$ \\
      \]\\
      subcat & \<\[
        loc|cat & \[
          head & \[ \asort{noun}
            cas & nom \\
            num & sg \\
          \]\\
          subcat & \<\> \\
        \] \\
      \]
      \> \\
    \] \\
  \]
\end{avm}

\begin{avm}
  \[\asort{word}
    phon & \<\rm\it meistens \> \\
    loc & \[
      cat & \[
        head & \[
          mod & \@{100} \[
            loc & \[
              cat|head & \[ \asort{verb} \] \\
              cont & \@{1} \\
            \] \\
          \] \\
        \] \\
        subcat & \<\> \\
      \] \\
      cont & \[\asort{psoa}
        restr & \[\asort{most-of-the-time-rel}
          pso-arg & \@{1} \\
        \] \\
      \] \\
    \] \\
  \]
\end{avm}

\subsection{Spuren}

\subsubsection{Verbspur}

\begin{avm}
  \[ phon & \<\> \\
    loc  & \@{100} \[cat|head|dsl & \@{100}\] \\
  \]
\end{avm}

\subsubsection{Argumentspur}

\begin{avm}
	\[\asort{word}
		phon & \<\> \\
                loc  & \orongsch{\@{200}} \\
                nonloc & \[slash & \<\orongsch{\@{200}} \>\]
	\]
\end{avm}

\section{Aufbau des Satzes}

\subsection{Struktur mit Semantik-Skizze}

\scalebox{0.8}{%
\begin{avm}
  \[ \asort{hd-filler-phr}
    phon & \<\rm\it Meistens filmt \_ Dan \_\> \\
    loc & \[ cat & \[
      head & \tuerkis{\@{80}} \\
      subcat & \<\> \\
    \] \\
    \violett{cont} & \violett{\@{1000}} \\
  \] \\
    nonloc|slash & \<\> \\
    filler-dtr & \[
      phon & \<\rm\it meistens \> \\
      loc & \orongsch{\@{200} \[
        cat & \[
          head & \[
            mod & \blau{\@{100} \[
              loc|cat|head & \[ \asort{verb} \] \\
            \]} \\
          \] \\
          subcat & \<\> \\
        \] \\
        \violett{cont} & \violett{\@{1000}} \\
      \]} \\
      nonloc|slash & \<\> \\
    \] \\
    hd-dtr & \[ \asort{hd-arg-phr}
      phon & \<\rm\it filmt \_ Dan \_ \> \\
      loc & \[ cat & \[
        head & \tuerkis{\@{80}} \\
        subcat & \<\> \\
      \] \\
      \violett{cont} & \violett{\@{1000}} \\
    \] \\
      nonloc|slash & \<\orongsch{\@{200}}\> \\
      hd-dtr & \[\asort{v1-lr}
          phon & \<\rm\it filmt \_\_\_\>\\
          loc & \[ cat & \[
            head & \tuerkis{\@{80} \[ \asort{verb}
              vform & fin \\
              initial & $+$ \\
            \]} \\
            subcat & \<\braun{\@{50} \[loc & \[ cat & \[
            head|dsl & \rot{\@{1}} \\
            subcat & \<\> \\
          \]\] \\
          \violett{cont} & \violett{\@{1000}} \\
          \]}\> \\
          \] \\
          \violett{cont} & \violett{\@{1000}} \\
          \] \\ 
          nonloc|slash & \<\> \\
          lex-dtr & \[
            phon & \<\rm\it filmt\> \\
            loc & \rot{\@{1} \[ cat & \[
              head & \gruen{\@{10}} \\
              subcat & \<\[
                \pinku{\@{90} loc|cat} & \pinku{\[
                head & \[ \asort{noun}
                  cas & nom \\
                  num & sg \\
                \]\\
                subcat & \<\> \\
              \]} \\
            \]\>
          \] \\
          \violett{cont} & \violett{\@{1002} \[ \asort{psoa}
            restr & \<\[\asort{film-rel} 
              agent & \@{1003} \\
            \]\> \\
          \]}\\
        \]} \\
        \]\\
      \] \\
      nhd-dtr & \braun{\@{50} \[ \asort{hd-mod-phr}
        phon & \<\rm\it \_ Dan \_ \> \\
        loc & \[ cat|head & \gruen{\@{10}} \\
          \violett{cont} & \violett{\@{1000}\[
            restr & \<\[\asort{most-of-the-time-rel}
            psoa-arg & \@{1002} \\
          \]\>\]}\\
        \] \\
        nonloc|slash & \<\orongsch{\@{200}}\> \\
        nhd-dtr & \[
          phon & \<\> \\
          loc & \orongsch{\@{200}\[
            \violett{cont} & \violett{\@{1000}} \\
          \]}  \\
          nonloc|slash & \<\orongsch{\@{200}}\> \\
        \]\\
        hd-dtr & \blau{\@{100} \[ \asort{hd-arg-phr}
          phon & \<\rm\it Dan \_ \> \\
          loc & \[ cat & \[
            head & \gruen{\@{10} \[ \asort{verb}
              vform & fin \\
              initial & $-$ \\
            \]}\\
            subcat & \<\> \\
          \] \\
          \violett{cont} & \violett{\@{1002}\[\asort{psoa}
            restr & \< \[ \asort{film-rel}
              agent & \@{1003}
            \]\> \\
          \]} \\
        \] \\
          nonloc|slash & \<\> \\
          hd-dtr & \[ \asort{word}
            phon & \<\> \\
            loc & \rot{\@{1} \[
              cat & \[
                head & \gruen{\@{10} \[ \asort{verb}
                  dsl & \@{1} \\
                \]}\\
                \violett{cont} & \violett{\@{1002}} \\
              \]
            \]} \\
          nonloc|slash & \<\> \\
          \] \\
          nhd-dtr & \pinku{\@{90} \[ \asort{word}
            phon & \<\rm\it Dan\> \\
            loc & \[ cat & \[
              head & \[ \asort{noun}
                case & nom \\
                num & sg \\
              \]\\
              subcat & \<\> \\
            \] \\
            \violett{cont} & \violett{\[
              index & \@{1003} \[
                indiv & dan \\
              \] \\
            \]} \\
          \]\\
          nonloc|slash & \<\> \\
          \]} \\
        \]} \\
      \]}
    \]
  \]
\end{avm}%
}

\newpage

\subsection{Erläuterungen zur Semantik}

Im Baum sind \violett{alle \textsc{cont}-Werte violett.}\\

Wir gehen hier von unten nach oben durch die Struktur und motivieren die Semantik.
Es wird jeweils der komplette Pfad durch die Struktur und der \textsc{phon}-Wert der Konstituente angegeben, damit klar ist, welcher \textsc{cont} gemeint ist.

\subsubsection{\textsc{hd-dtr|nhd-dtr|hd-dtr|nhd-dtr} (\textit{Dan})}

Das ist die lexikalische Semantik eines Eigennamens. 
Er bringt einen Index mit, der konstant auf eine Person verweist.

\subsubsection{\textsc{hd-dtr|nhd-dtr|hd-dtr|hd-dtr} (Verbspur)}

Die Verbspur bekommt ihre Semantik über \textsc{dsl} von ganz oben aus der \textsc{lex-dtr} des bewegten Verbs (\textsc{hd-dtr|hd-dtr|lex-dtr}), s.\ \mybox{1002}.
Dort sehen wir, dass die Semantik ein \textsc{psoa} ist, der als \textsc{restr} nur eine \textit{film-rel} mitbringt, die ein Agens hat.

\subsubsection{\textsc{hd-dtr|nhd-dtr|hd-dtr} (\textit{Dan \_})}\label{sec:danvspur}

Die Semantik ist wieder die \mybox{1002} dank des Semantikprinzips (Teil 1 für \textsc{head-arg-phrase}).
Hier wird in der Darstellung die Identifikation des rollentragenden Index \mybox{1003} mit der Agens-Rolle des Verbs angezeigt.
(Aber wie üblich Obacht bitte: Überall wo \mybox{1002} steht, steht dank Unifikation und Strukturteilung dasselbe.)\\

Hinweis: Ich habe im Seminar zum Schluss etwas stark gesagt, dass über \textsc{dsl} die lexikalische Semantik des Verbs protokolliert werde.
Das stimmt auch, aber natürlich werden die normalen Rollen mit Indizes unifiziert wie in jeder Valenzabbindung, und das ist dank Strukturteilung dann auch unter \textsc{dsl} sichtbar.
Was in \textsc{dsl} nicht überschrieben wird ist die prinzipielle Semantik des Verbs\slash der Verbspur, so wie es hier eine Etage höher durch den Modifizierer geschieht.
Das passiert dann zwar in der Phrasensemantik, nicht aber in \textsc{dsl}.

\subsubsection{\textsc{hd-dtr|nhd-dtr|nhd-dtr} (Adjunktspur)}\label{sec:aspur}

Hier steht die Semantik, die weiter oben (beim Filler) mit \mybox{1000} genau gezeigt wird.
Weil von hier aufwärte der \textsc{local}-Wert \mybox{200} weitergegeben wird, bis er in Form des Fillers gefüllt wird, ist hier die komplette Syntax-Semantik von \textit{meistens} verfügbar.
Beachten Sie auch den Lexikoneintrag für \textit{meistens} in Abschnitt~\ref{sec:lex}.
Was genau passiert, wird eine Etage höher gezeigt:

\subsubsection{\textsc{hd-dtr|nhd-dtr} (\_ Dan \_)}

Über sein \textsc{mod}-Merkmal verschafft sich \textit{wenigstens} Zugriff auf die Semantik des Modifikans (s.\ Lexikoneintrag), in diesem Fall \textit{Dan \_} -- also \mybox{1002}.
Diese nimmt es als \textsc{psoa-arg}, weil man nicht vernünftig die Menge der \textit{meistenen} Dinge mit denen der \textit{Dan filmt}-Dinge schneiden kann.
Das hat bei solchen relativ quantifizierenden Zeit-Adverbialen keinen Sinn.\\

Weil es sich um eine \textsc{hd-mod-phr} handelt, kommt die Semantik vom Nicht-Kopf (Semantikprinzip Teil 2), also \mybox{1000} von der Adunktspur.

\subsubsection{\textsc{hd-dtr|hd-dtr|lex-dtr} (\textit{filmt})}

Das ist einfach die lexikalische Semantik von \textit{filmt}.
S.\ Lexikoneintrag und Abschnitt~\ref{sec:danvspur}.\\

Da diese Semantik Teil des \textsc{local}-Werts \mybox{1} ist, der über \textsc{dsl} zur Spur gelangt, ist sie beim Aufbau der gespurten VP verfügbar.
Dass dies gelingt, liegt wesentlich daran, dass die \textit{v1-lr} auf ihrer \textsc{subcat} den \textsc{dsl} der gespurten VP mit dem \textsc{local} ihrer \textsc{lex-dtr} identifiziert.

\subsubsection{\textsc{hd-dtr|hd-dtr} (\textit{filmt \_\_\_})}\label{sec:v1lr}

Die \textsc{v1-lr} sagt explizit (s.\ Musterlösung 4), dass sie ihren \textsc{cont} von dem der gespurten VP übernimmt.
Das geschieht hier über die \mybox{1000}.

\subsubsection{\textsc{hd-dtr} (\textit{filmt \_ Dan \_})}

Dies Semantik in \mybox{1000} wird über Teil 1 des Semantikprinzips von der Kopftochter (s.\ Abschnitt~\ref{sec:v1lr}) übernommen.

\subsubsection{\textsc{filler-dtr} (\textit{meistens})}

Hier wird die Semantik des lexikalischen Eintrags für \textit{meistens} (s.\ Abschnitt~\ref{sec:lex}) mit der der Adjunktspur unifiziert.
Das wurde bereits in Abschnitt~\ref{sec:aspur} besprochen.
Das Schema für die \textit{hd-filler-phr} sagt, dass der \textsc{local}-Wert des Fillers mit der Information auf \textsc{slash} unifiziert wird.
Das ist \mybox{200}, und der \textsc{cont} \mybox{1000} ist Teil davon.

\subsubsection{\textit{hd-filler-phrase} (\textit{Meistens filmt \_ Dan \_})}

Der gesamte Satz hat dank Semantikprinzip die Semantik der Kopftochter, also \mybox{1000}.
Ausbuchstabiert sieht die Semantik des Satzes so aus:\\

\begin{avm}
  \[\asort{psoa}
    restr & \<\[\asort{most-of-the-time-rel}
      psoa-arg & \[\asort{psoa}
        restr & \<\[\asort{film-rel}
          agent & \[\asort{index}
            indiv & dan \\
          \]\\
        \]\>
      \]
    \]\>
  \]
\end{avm}

\Zeile

Umgangssprachlich heißt das, dass es eine Situation gibt, in der es meistens der Fall ist, dass es eine \textit{Filmen}-Situation gibt, in der Dan das Agens ist.
Das passt ja auch ziemlich genau zur Bedeutung des Satzes.

\end{document}
