\documentclass[10pt,a4paper]{article}

\usepackage[margin=2cm]{geometry}

\usepackage{booktabs}
\usepackage{color}
\usepackage{soul}
\usepackage[linecolor=gray,backgroundcolor=yellow!50,textsize=tiny]{todonotes}
\usepackage[linguistics]{forest}
\usepackage{styles/my-gb4e-slides}

\usepackage{avm}

\usepackage[maxbibnames=99,
  maxcitenames=2,
  uniquelist=false,
  backend=biber,
  doi=false,
  url=false,
  isbn=false,
  bibstyle=biblatex-sp-unified,
  citestyle=sp-authoryear-comp]{biblatex}

\definecolor{rot}{rgb}{0.7,0.2,0.0}
\newcommand{\rot}[1]{\textcolor{rot}{#1}}
\definecolor{blau}{rgb}{0.1,0.2,0.7}
\newcommand{\blau}[1]{\textcolor{blau}{#1}}
\definecolor{gruen}{rgb}{0.0,0.7,0.2}
\newcommand{\gruen}[1]{\textcolor{gruen}{#1}}
\definecolor{grau}{rgb}{0.6,0.6,0.6}
\newcommand{\grau}[1]{\textcolor{grau}{#1}}
\definecolor{orongsch}{RGB}{255,165,0}
\newcommand{\orongsch}[1]{\textcolor{orongsch}{#1}}
\definecolor{tuerkis}{RGB}{63,136,143}
\definecolor{braun}{RGB}{108,71,65}
\newcommand{\tuerkis}[1]{\textcolor{tuerkis}{#1}}
\newcommand{\braun}[1]{\textcolor{braun}{#1}}

\forestset{
  decide/.style={draw, chamfered rectangle, inner sep=2pt},
  finall/.style={rounded corners, fill=gray, text=white},
  intrme/.style={draw, rounded corners},
  yes/.style={edge label={node[near end, above, sloped, font=\scriptsize]{Ja}}},
  no/.style={edge label={node[near end, above, sloped, font=\scriptsize]{Nein}}}
}

\author{Roland Schäfer}
\title{HPSG: Musterlösung 1}
\date{\today}

\usepackage{fontspec}
\defaultfontfeatures{Ligatures=TeX,Numbers=OldStyle, Scale=MatchLowercase}
\setmainfont{Linux Libertine O}
\setsansfont{Linux Biolinum O}

\avmfont{\sc}
\avmsortfont{\it}
\avmvalfont{\it}

\pagestyle{empty}

\begin{document}

\maketitle

\thispagestyle{empty}

\section{Material}

\noindent Es geht hier ausschließlich um Satz (\ref{ex:satz}).

\begin{exe}
  \ex Schläft das Kind?\label{ex:satz}
\end{exe}

\section{Lexikoneinträge}

\subsection{Verb}

\noindent\begin{avm}
  \[
    phon & schläft \\
    head & \[ \asort{verb}
      vform & fin
    \]\\
    subcat & \< \[
      head & \[ \asort{noun}
        case & nom 
      \]\\
      subcat & \<\>
    \] \>\\
  \]
\end{avm}

\subsection{Substantiv\slash Nomen}

\noindent\begin{avm}
  \[
    phon & Kind \\
    head & \[ \asort{noun}
      case & \@1 nom 
    \]\\
    subcat & \< \[
      head & \[ \asort{det}
        case & \@1
      \] \\
      subcat & \<\>
    \]\>
  \]
\end{avm}

\subsection{Determinierer}

\noindent\begin{avm}
  \[
    phon & das \\
    head & \[ \asort{det}
      case & nom 
    \]\\
    subcat & \<\>
  \]
\end{avm}

\newpage

\section{Schemata und Prinzipien}

\noindent Zusätzlich zu den Lexikoneinträgen brauchen wir nur zwei sehr einfache Regularitäten, um den gesamten Satz zu mo\-dellieren.
Das Kopf-Argument-Schema sagt uns, wie Phrasen bei Valenzabbindung gebildet werden, und das Kopfmerkmalprinzip sagt uns, wie Phrasen ihre kategorialen Werte erhalten (also ob sie ein NP, VP usw.\ sind).
Wie die \textsc{phon}-Werte von Phrasen gebildet werden, haben wir noch nicht besprochen.
Wir nehmen daher informell an, dass das schon irgendwie funktionieren wird.

\subsection{Kopf-Argument-Schema}

\textit{hd-arg-phrase} $\Rightarrow$
\begin{avm}
  \[
    subcat & \@1 \\
    hd-dtr|subcat & \@1 $\oplus$ \< \@2 \> \\
    nhd-dtr & \@2
  \]
\end{avm}

\subsection{Kopfmerkmalprinzip}

\noindent \textit{headed-phrase} $\Rightarrow$
\begin{avm}
  \[
    head & \@1 \\
    hd-dtr|head & \@1
  \]
\end{avm}


\section{Phrasenbildung}

\noindent Es werden hier jeweils die Lexikoneinträge bzw.\ Tochterphrasen verbatim in die Phrasen kopiert.
Farblich werd dann die Beiträge des \blau{Kopf-Argument-Schemas} (blau) und des \rot{Kopfmerkmalprinzips} (rot) hinzugefügt.
Außerdem werden selektiv \gruen{Strukturteilungen} (gruen) markiert, die sich im Rahmen der Phrasenbildung aus Lexikoneinträgen ergeben.
Konkret ist das hier Kongruenz in der Nominalphrase.
Die \grau{\textsc{phon}-Werte} von Phrasen nehmen wir als gegeben an, haben aber eigentlich noch keine Regularität eingeführt, die sie zusammensetzt.
Man schreibt das in HPSG aus guten Gründen nicht einfach in das Kopf-Argument-Schema mit hinein, sondern trennt Dominanz (Phrasenbildung) und Präzedenz (Reihenfolge).

\subsection{Die Nominalphrase}

\begin{avm}
  \[ \asort{\blau{hd-arg-phrase}}
    \grau{phon} & \grau{das Kind} \\
    \rot{head} & \rot{\@2} \\
    \blau{subcat} & \blau{\<\>} \\
    \blau{hd-dtr} & \[
        phon & Kind \\
        head & \rot{\@2} \[ \asort{noun}
          case & \gruen{\@1 nom}
        \]\\
        subcat & \< \blau{\@3} \[
          head & \[ \asort{det}
            case & \gruen{\@1}
          \] \\
          subcat & \<\>
        \]\>
      \] \\
      \blau{nhd-dtr} & \blau{\@3} \[
        phon & das \\
        head & \[ \asort{det}
          case & \gruen{\@1 nom}
        \]\\
        subcat & \<\>
      \]
  \]
\end{avm}

\subsection{Verbalphrase\slash Satz}

\begin{avm}
  \[ \asort{\blau{hd-arg-pghrase}}
    \grau{phon} & \grau{schläft das Kind} \\
    \rot{head} & \rot{\@4} \\
    \blau{subcat} & \blau{\<\>} \\
    \blau{hd-dtr} & \[
      phon & schläft \\
      head & \rot{\@4} \[ \asort{verb}
        vform & fin
      \]\\
      subcat & \< \blau{\@5} \[
        head & \[ \asort{noun}
          case & nom 
        \]\\
        subcat & \<\>
      \] \>\\
    \] \\
    \blau{nhd-dtr} & \blau{\@5} \[ \asort{hd-arg-phrase}
      phon & das Kind \\
      head & \@2 \\
      subcat & \<\> \\
      hd-dtr & \[
          phon & Kind \\
          head & \@2 \[ \asort{noun}
            case & \@1 nom 
          \]\\
          subcat & \< \@3 \[
            head & \[ \asort{det}
              case & \@1
            \] \\
            subcat & \<\>
          \]\>
        \] \\
        nhd-dtr & \@3 \[
          phon & das \\
          head & \[ \asort{det}
            case & \@1 nom 
          \]\\
          subcat & \<\>
        \]
    \]
  \]
\end{avm}

\section{Satzanalyse mit Semantik}

\subsection{Verb}

\noindent Hier nehmen wir den Semantik-Trick (nicht aus dem Buch) an, bei dem das Verb die \textsc{restr}-Liste seiner Argumente auf der eigenen \textsc{restr}-Liste vereinnahmt.
Um zu illustrieren, dass die Labels für die Strukturteilung in jeder Struktur neu vergeben werden, wurde hier auf arbiträre Zähler (100 und 101) zurückgegriffen.\\

\noindent\begin{avm}
  \[
    phon & schläft \\
    head & \[ \asort{verb}
      vform & fin
    \]\\
    \orongsch{cont} & \orongsch{\[
      restr & \<\[ \asort{schlafen-rel}
        experiencer & \@{100} \\
      \]\> $\oplus$ \@{101} \\
    \]} \\
    subcat & \< \[
      head & \[ \asort{noun}
        case & nom 
      \]\\
      subcat & \<\>\\
      \orongsch{cont} & \orongsch{\[
        index & \@{100} \\
        restr & \@{101} \\
      \]} \\
    \] \>\\
  \]
\end{avm}

\subsection{Semantikprinzip}

\noindent \textit{headed-phrase} $\Rightarrow$
\begin{avm}
  \[
    cont & \@1 \\
    hd-dtr|cont & \@1
  \]
\end{avm}


\subsection{Satz}

\noindent Die allgemeine \orongsch{Semantik} ist orange markiert.
Der spezifische Beitrag des \tuerkis{Semantikprinzips} ist türkis markiert.\\

\begin{avm}
  \[ \asort{hd-arg-pghrase}
    phon & schläft das Kind \\
    head & \@4 \\
    \orongsch{cont} & \tuerkis{\@6} \\
    subcat & \<\> \\
    hd-dtr & \[
      phon & schläft \\
      head & \@4 \[ \asort{verb}
        vform & fin
      \]\\
      \orongsch{cont} & \tuerkis{\@6} \orongsch{\[
        restr & \<\[ \asort{schlafen-rel}
          experiencer & \@8 \\
        \]\> $\oplus$ \@9 \\
      \]} \\
      subcat & \< \@5 \[
        head & \[ \asort{noun}
          case & nom 
        \]\\
        subcat & \<\>\\
        \orongsch{cont} & \orongsch{\[
          index & \@8 \\
          restr & \@9 \\
        \]} \\
      \] \>\\
    \] \\
    nhd-dtr & \@5 \[ \asort{hd-arg-phrase}
      phon & das Kind \\
      head & \@2 \\
      \orongsch{cont} & \tuerkis{\@7} \\
      subcat & \<\> \\
      hd-dtr & \[
          phon & Kind \\
          head & \@2 \[ \asort{noun}
            case & \@1 nom 
          \]\\
          \orongsch{cont} & \tuerkis{\@7} \orongsch{\[
            index & \@8 \[
              num & sg \\
              per & 3 \\
              gen & n
            \]\\
            restr & \@9 \< \[ \asort{kind-rel}
              instance & \@8 \\
            \] \>  \\
          \] } \\
          subcat & \< \@3 \[
            head & \[ \asort{det}
              case & \@1
            \] \\
            subcat & \<\>
          \]\>
        \] \\
        nhd-dtr & \@3 \[
          phon & das \\
          head & \[ \asort{det}
            case & \@1 nom 
          \]\\
          subcat & \<\>
        \]
    \]
  \]
\end{avm}

\section{Zusatz: Lexioneintrag für \textit{gibt}}

\subsection{Konform zum Buch}

\begin{avm}
  \[
  phon & gibt \\
  head & \[ \asort{verb}
    vform & fin 
  \]\\
  cont & \[ 
    restr & \[ \asort{geben-rel}
      agent & \@{200} \\
      theme & \@{201} \\
      goal & \@{202} \\
    \]
  \]\\
  subcat & \<
  \[
    head & \[\asort{noun}
      case & nom \\
    \]\\
    subcat & \<\> \\
    cont|ind & \@{200}\\
  \]
  ,
  \[
    head & \[\asort{noun}
      case & acc \\
    \]\\
    subcat & \<\> \\
    cont|ind & \@{201}
  \]
  ,
  \[
    head & \[\asort{noun}
      case & dat \\
    \]\\
    subcat & \<\> \\
    cont|ind & \@{202}
  \]
  \>
  \]
\end{avm}


\subsection{Version mit Semantiktrick}

\begin{avm}
  \[
  phon & gibt \\
  head & \[ \asort{verb}
    vform & fin 
  \]\\
  cont & \[ 
    restr & \<
    \[ \asort{geben-rel}
      agent & \@{200} \\
      theme & \@{201} \\
      goal & \@{202}  \\
    \]
    \>
    $\oplus$ \@{300} 
    $\oplus$ \@{301} 
    $\oplus$ \@{302} 
  \]\\
  subcat & \<
  \[
    head & \[\asort{noun}
      case & nom \\
    \]\\
    subcat & \<\> \\
    cont & \[
      index & \@{200} \\
      restr & \@{300} \\
    \]
  \]
  ,
  \[
    head & \[\asort{noun}
      case & acc \\
    \]\\
    subcat & \<\> \\
    cont & \[
      index & \@{201} \\
      restr & \@{301} \\
    \]
  \]
  ,
  \[
    head & \[\asort{noun}
      case & dat \\
    \]\\
    subcat & \<\> \\
    cont & \[
      index & \@{202} \\
      restr & \@{302} \\
    \]
  \]
  \>
  \]
\end{avm}



\end{document}
