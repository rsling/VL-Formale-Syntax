\documentclass[10pt,a3paper]{article}

\usepackage[margin=2cm]{geometry}

\usepackage{booktabs}
\usepackage{color}
\usepackage{soul}
\usepackage[linecolor=gray,backgroundcolor=yellow!50,textsize=tiny]{todonotes}
\usepackage[linguistics]{forest}
\usepackage{styles/my-gb4e-slides}

\usepackage{avm}

\usepackage[maxbibnames=99,
  maxcitenames=2,
  uniquelist=false,
  backend=biber,
  doi=false,
  url=false,
  isbn=false,
  bibstyle=biblatex-sp-unified,
  citestyle=sp-authoryear-comp]{biblatex}

\addbibresource{register.bib}

\newcommand{\eg}{e.\,g.}
\newcommand{\Eg}{E.\,g.}

\definecolor{rot}{rgb}{0.7,0.2,0.0}
\newcommand{\rot}[1]{\textcolor{rot}{#1}}
\definecolor{blau}{rgb}{0.1,0.2,0.7}
\newcommand{\blau}[1]{\textcolor{blau}{#1}}
\definecolor{gruen}{rgb}{0.0,0.7,0.2}
\newcommand{\gruen}[1]{\textcolor{gruen}{#1}}
\definecolor{grau}{rgb}{0.6,0.6,0.6}
\newcommand{\grau}[1]{\textcolor{grau}{#1}}
\definecolor{orongsch}{RGB}{255,165,0}
\newcommand{\orongsch}[1]{\textcolor{orongsch}{#1}}
\definecolor{tuerkis}{RGB}{63,136,143}
\definecolor{braun}{RGB}{108,71,65}
\newcommand{\tuerkis}[1]{\textcolor{tuerkis}{#1}}
\newcommand{\braun}[1]{\textcolor{braun}{#1}}

\newcommand*{\mybox}[1]{\framebox{#1}}

\forestset{
  decide/.style={draw, chamfered rectangle, inner sep=2pt},
  finall/.style={rounded corners, fill=gray, text=white},
  intrme/.style={draw, rounded corners},
  yes/.style={edge label={node[near end, above, sloped, font=\scriptsize]{Ja}}},
  no/.style={edge label={node[near end, above, sloped, font=\scriptsize]{Nein}}}
}

\author{Roland Schäfer}
\title{HPSG: Musterlösung 3}
\date{\today}

\usepackage{fontspec}
\defaultfontfeatures{Ligatures=TeX,Numbers=OldStyle, Scale=MatchLowercase}
\setmainfont{Linux Libertine O}
\setsansfont{Linux Biolinum O}

\avmfont{\sc}
\avmsortfont{\it}
\avmvalfont{\it}

\pagestyle{empty}

\begin{document}

\maketitle

\thispagestyle{empty}

\section{Material}

\noindent Es geht hier ausschließlich um Satz (\ref{ex:satz}).

\begin{exe}
  \ex Die Freude wegen der Geschenke macht glücklich.\label{ex:satz}
\end{exe}

\section{Eintrag für \textit{wegen}}

\subsection{Ohne Semantik (wie an der Tafel)}

Statt einen Typ wie \textit{ne} oder \textit{nelist} (für \textit{non-empty list}) zu bemühen, habe ich hier unter \textsc{cat|head|mod|cat|subcat} die Beschreibung eines Determinierers kodiert.
Das würde beides für unsere Zwecke ungefähr gleich gut funktionieren.
Für Nomina mit komplexerer Valenz bzw.\ relationale Nomina (wie \textit{Eroberung}) müsste man überlegen, ob eine Version wie hier in der Musterlösung nicht besser ist.\\

\begin{avm}
  \[ \asort{word}
    phon & \<\rm\it wegen\> \\
    cat & \[
      head & \[
        \asort{prep}
        mod | cat & \[
            head & \[ \asort{noun} \] \\
            subcat & \< \[ cat | head & \[ \asort{det} \]\] \>
          \]
      \]  \\
      subcat & \<\[
        cat \[
          head & \[ \asort{noun}
            case & gen
          \]\\
          subcat & \<\>\\
        \]
      \]\>\\
    \] \\
  \]
\end{avm}


\subsection{Mit Semantik}

Ich habe im Seminar die Indices, die hier \mybox{1} und \mybox{3} numeriert sind, verdreht.
Hier ist es jetzt richtig.
Überlegen Sie, was das Problem gewesen wäre.
Es tritt auf, wenn Semantik der gesamten Phrase ((\textit{die}) \textit{Freude wegen der Geschenke}) zusammengebaut wird bzw.\ wenn sich diese NP mit dem Verb verbindet.\\

\begin{avm}
  \[ \asort{word}
    phon & \<\rm\it wegen\> \\
    cat & \[
      head & \[
        \asort{prep}
        mod & \[
          cat & \[
            head & \[ \asort{noun} \] \\
            subcat & \< \[ cat | head & \[ \asort{det} \]\] \>
          \] \\
          \rot{cont} & \rot{\[
            index & \@1 \\
            restr & \@2 \\
          \]}
        \]
      \]  \\
      subcat & \<\[
        cat & \[
          head & \[ \asort{noun}
            case & gen
          \]\\
          subcat & \<\>\\
        \] \\
        \rot{cont} & \rot{\[
          index & \@3 \\
          restr & \@4 \\
        \]}\\
      \]\>\\
    \] \\
    \rot{cont} & \rot{\[
      ind & \@1 \\
      restr & \< 
      \[ \asort{cause-rel}
        cause & \@3 \\
        effect & \@1  \\
      \]
      \> $\oplus$ \@2 $\oplus$ \@4 \\
    \]}
  \]
\end{avm}

\newpage

\subsection{PP}\label{sec:pp}

\noindent Verglichen mit dem Lexikoneintrag ist hier die Information der \textsc{subcat} von \textit{wegen} in der Darstellung zu \textsc{nhd-dtr} verschoben (s.\ \mybox{11}).
Was unter 11 steht, muss aber unifizieren mit der Information auf der \textsc{subcat} des Lexikoneintrags.
  Überprüfen Sie, ob das so ist.\\

\begin{avm}
  \[
    \asort{hd-arg-phrase}
    phon & \<\rm\it wegen der Geschenke\> \\
    cat & \[
      head & \orongsch{\@{10}} \\
      subcat & \<\> \\
    \] \\
    cont & \blau{\@{20}} \\
    hd-dtr & \[ \asort{word}
      phon & \<\rm\it wegen\> \\
      cat & \[
        head & \orongsch{\@{10}} \[
          \asort{prep}
          mod & \[
            cat & \[
              head & \[ \asort{noun} \] \\
              subcat & \< \[ cat | head & \[ \asort{det} \]\] \>
            \] \\
            cont & \[
              index & \@1 \\
              restr & \@2 \\
            \]
          \]
        \]  \\
        subcat & \< \gruen{\@{11}} \>\\
      \] \\
      cont & \blau{\@{20}} \[
        ind & \@1 \\
        restr & \< 
        \[ \asort{cause-rel}
          cause & \@3 \\
          effect & \@1  \\
        \]
        \> $\oplus$ \@2 $\oplus$ \@4 \\
      \]
    \] \\
    nhd-dtr & \gruen{\@{11}} \[ \asort{hd-arg-phrase}
      phon & \<\rm\it der Geschenke \> \\
      cat & \[ head & \[ \asort{noun}
        case & gen \\
        num & pl \\
        \] \\
        subcat & \<\> \\
      \] \\
      cont & \[
            index & \@3 \\
            restr & \@4 \< \[ \asort{geschenke-rel}
              instance & \@3 \\
            \]\>\\
          \]\\ \\
    \]
  \]
\end{avm}

\newpage

\section{$\bar{N}$ mit PP-Modifikator}

\noindent Auch hier wurde wieder in der Darstellung verschoben.
Statt unter dem \textsc{mod} Merkmal von \textit{wegen der Geschenke} ist die Beschreibung der modifizierten Phrase komplett in \textsc{hd-dtr} (\textit{Freude}) verschoben (s.\ \mybox{30}).
Was dort steht, muss aber mit dem \textsc{cat|head|mod} von \textit{wegen der Geschenke} (s.\ \ref{sec:pp}) unifizieren.
Überprüfen Sie, ob das so ist.\\

\begin{avm}
  \[ \asort{hd-mod-phrase}
    phon & \<\rm\it Freude wegen der Geschenke\> \\
    cat & \[
      head & \braun{\@{31}} \\
      subcat & \braun{\@{32}} \\
    \]\\
    cont & \blau{\@{20}} \\
    hd-dtr & \tuerkis{\@{30}} \[ \asort{word}
      phon & \<\rm\it Freude \> \\
      cat & \[
        head & \braun{\@{31}} \[ \asort{noun} 
          case & nom \\
          num & sg \\
        \] \\
        subcat & \braun{\@{32}} \< \[ cat | head & \[ \asort{det} \]\] \>
      \] \\
      cont & \[
        index & \@1 \\
        restr & \@2 \\
      \]
    \] \\
    nhd-dtr & \[
      \asort{hd-arg-phrase}
      phon & \<\rm\it wegen der Geschenke\> \\
      cat & \[
        head & \orongsch{\@{10}} \\
        subcat & \<\> \\
      \] \\
      cont & \blau{\@{20}} \\
      hd-dtr & \[ \asort{word}
        phon & \<\rm\it wegen\> \\
        cat & \[
          head & \orongsch{\@{10}} \[
            \asort{prep}
            mod & \tuerkis{\@{30}} 
          \]  \\
          subcat & \< \gruen{\@{11}} \>\\
        \] \\
        cont & \blau{\@{20}} \[
          ind & \@1 \\
          restr & \< 
          \[ \asort{cause-rel}
            cause & \@3 \\
            effect & \@1  \\
          \]
          \> $\oplus$ \@2 $\oplus$ \@4 \\
        \]
      \] \\
      nhd-dtr & \gruen{\@{11}} \[ \asort{hd-arg-phrase}
        phon & \<\rm\it der Geschenke \> \\
        cat & \[ head & \[ \asort{noun}
          case & gen \\
          num & pl \\
          \] \\
          subcat & \<\> \\
        \] \\
        cont & \[
              index & \@3 \\
              restr & \@4 \< \[ \asort{geschenke-rel}
                instance & \@3 \\
              \]\>\\
            \]\\ \\
      \]
    \] 
  \]
\end{avm}

\section{Spezifikation}

\subsection{Lexikoneintrag einfacher Determinierer}

Hier brauchen wir \textsc{spec} eigentlich gar nicht.\\

\begin{avm}
  \[ \asort{word}
    phon & \<\rm\it die\> \\
    cat & \[
      head & \[ \asort{det} 
        spec & none \\
      \]\\
    \] \\
  \]
\end{avm}

\subsection{Lexikoneintrag \textit{Ischariot}}

\noindent Der Typ \textit{proper-name} sei ein Untertyp von \textit{noun}.\\

\noindent Konstante Individuen wie Ischariot sollten nicht über eine Relation identifiziert werden.
Ich gehe hier davon aus, dass ein Eigenname einen \textbf{konstanten Index} beisteuert, der fest auf ein \textsc{indiv(idual)} verweist.
Auch das ist mal wieder etwas fürs Semantik-Seminar.\\

\begin{avm}
  \[ \asort{stem}
    phon & \<\rm\it Ischariot \> \\
    cat & \[
      head & \[ \asort{proper-name}
      \]\\
    \] \\
    cont & \[
      index & \@1 \[
        gen & masc \\
        indiv & ischariot \\
      \]\\
    \]\\
  \]
\end{avm}

\subsection{DLR für Eigennamengenitiv}

\noindent Ich nehme hier keine Besitz-Relation an, sondern eine allgemeinere \textit{attribution-rel}.\\

\begin{avm}
  \[ \asort{proper-name-genitive-lex-rule}
    phon & \@1 $\oplus$ \<\rm\it s\> \\
    cat & \[
      head & \[ \asort{det}
        case & gen \\
        spec & \[ cont|ind & \@3 \]\\
      \] \\
    \]  \\
    cont & \[
      restr & \< \[ \asort{attribution-rel}
        attribute & \@2 \\
        attributed & \@3 \\
      \]\> \\
    \] \\
    lex-dtr & \[ \asort{stem}
      phon & \@1 \\
      cat & \[
        head & \[ \asort{proper-name} \] \\
      \] \\
      cont|ind & \@2 \\
    \]\\
  \]
\end{avm}\\

\subsection{Anwendung der DLR auf \textit{Ischariot}}

Das Ergebnis kann dann als Determinierer verwendet werden.\\

\begin{avm}
  \[ \asort{proper-name-genitive-lex-rule}
    phon & \<\rm\it Ischariot, s\> \\
    cat & \[
      head & \[ \asort{det}
        case & gen \\
        spec & \[ cont|ind & \@3 \]\\
      \] \\
    \]  \\
    cont & \[
      restr & \< \[ \asort{attribution-rel}
        attribute & \@2 \\
        attributed & \@3 \\
      \]\> \\
    \] \\
    lex-dtr & \[ \asort{stem}
      phon & \<\rm\it Ischariot \> \\
      cat & \[
        head & \[ \asort{proper-name}
        \]\\
      \] \\
      cont & \[
        index & \@2 \[
          gen & masc \\
          indiv & ischariot \\
        \]\\
      \]\\
    \]\\
  \]
\end{avm}\\

\noindent Anmerkung: Es geht auch eleganter und ohne, dass man aus einem Nominalstamm einen Determinierer machen muss, was etwas unnatürlich ist.
Dann muss die \textsc{subcat} von Nomina entsprechend angepasst werden.
Überlegen Sie, wie das gehen würde.

\newpage

\subsection{NP mit modifizierender PP und Genitiv-\textsc{spec}}

Das \textsc{spec}-Prinzip stellt dann sicher, dass der pränominale Genitiv (\textit{Ischariots}) das \textsc{attributed}-Argument der \textit{attribution-rel} mit dem Index von \textit{Freude} identifizieren kann.\\

\noindent Wenn Sie genau hinsehen, können Sie feststellen, dass noch etwas fehlt.
Die Semantik der gesamten Phrase (also \mybox{20}) wird noch nicht korrekt modelliert.
Überlegen Sie, warum das so ist, und was man tun könnte, um das Problem zu lösen.\\

\begin{avm}
  \[ \asort{hd-arg-phrase}
    phon & \<\rm\it Ischariots Freude über die Geschenke \> \\
    cat & \[
      head & \@{31} \\
      subcat & \<\> \\
    \]\\
    cont & \@{20} \\
    nhd-dtr & \@{400} \[
      \asort{proper-name-genitive-lex-rule}
      phon & \<\rm\it Ischariots\> \\
      cat & \[
        head & \[ \asort{det}
          case & gen \\
          spec & \@{300} \\
        \] \\
      \]  \\
      cont & \[
        restr & \< \[ \asort{attribution-rel}
          attribute & \@{200} \\
          attributed & \@{1} \\
        \]\> \\
      \] \\
      lex-dtr & \[ \asort{stem}
        phon & \<\rm\it Ischariot \> \\
        cat & \[
          head & \[ \asort{proper-name}
          \]\\
        \] \\
        cont & \[
          index & \@{200} \[
            gen & masc \\
            indiv & ischariot \\
          \]\\
        \]\\
      \]\\
    \] \\
    hd-dtr & \@{300} \[ \asort{hd-mod-phrase}
      phon & \<\rm\it Freude wegen der Geschenke\> \\
      cat & \[
        head & \braun{\@{31}} \\
        subcat & \braun{\@{32} \< \@{400} \>} \\
      \]\\
      cont & \blau{\@{20}} \\
      hd-dtr & \tuerkis{\@{30}} \[ \asort{word}
        phon & \<\rm\it Freude \> \\
        cat & \[
          head & \braun{\@{31}} \[ \asort{noun} 
            case & nom \\
            num & sg \\
          \] \\
          subcat & \braun{\@{32}} 
        \] \\
        cont & \[
          index & \@1 \\
          restr & \@2 \\
        \]
      \] \\
      nhd-dtr & \[
        \asort{hd-arg-phrase}
        phon & \<\rm\it wegen der Geschenke\> \\
        cat & \[
          head & \orongsch{\@{10}} \\
          subcat & \<\> \\
        \] \\
        cont & \blau{\@{20}} \\
        hd-dtr & \[ \asort{word}
          phon & \<\rm\it wegen\> \\
          cat & \[
            head & \orongsch{\@{10}} \[
              \asort{prep}
              mod & \tuerkis{\@{30}} 
            \]  \\
            subcat & \< \gruen{\@{11}} \>\\
          \] \\
          cont & \blau{\@{20}} \[
            ind & \@1 \\
            restr & \< 
            \[ \asort{cause-rel}
              cause & \@3 \\
              effect & \@1  \\
            \]
            \> $\oplus$ \@2 $\oplus$ \@4 \\
          \]
        \] \\
        nhd-dtr & \gruen{\@{11}} \[ \asort{hd-arg-phrase}
          phon & \<\rm\it der Geschenke \> \\
          cat & \[ head & \[ \asort{noun}
            case & gen \\
            num & pl \\
            \] \\
            subcat & \<\> \\
          \] \\
          cont & \[
                index & \@3 \\
                restr & \@4 \< \[ \asort{geschenke-rel}
                  instance & \@3 \\
                \]\>\\
              \]\\ \\
        \]
      \] 
    \]
  \]
\end{avm}







\end{document}


\begin{avm}
  \[ \asort{word}
    phon & \<\rm\it \> \\
    cat & \[
      head & \[ \asort{}
      \]\\
      subcat & \<\> \\
    \] \\
    cont & \[
      index & \\
      restr & \<\> \\
    \]\\
  \]
\end{avm}



